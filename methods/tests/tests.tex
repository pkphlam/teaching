\documentclass{beamer}


\usetheme{default}
\usepackage{subfigure}
\usepackage{amsmath}
\usepackage{Sweave}
\usepackage{graphicx}
\usepackage{color}
\usepackage{amsfonts}
\usepackage{amssymb}
\usepackage{multicol}
\usepackage[all]{xy}
\usepackage{bm}


\author{Patrick Lam}
\title{Statistical Tests}
\date{}
%\date{December 8, 2008}

\begin{document}

\newcommand{\red}{\textcolor{red}}
\newcommand{\blue}{\textcolor{blue}}
\newcommand{\purple}{\textcolor{purple}}

\frame{\titlepage}

\begin{frame}
\frametitle{Outline}
\tableofcontents
\end{frame}

\section{Testing Independence}

\begin{frame}
\frametitle{Outline}
\tableofcontents[currentsection]
\end{frame}

\subsection{The Chi-Square Test for Independence}

\begin{frame}
\frametitle{The Chi-Square Test for Independence of Two Variables}
\pause
Suppose you have two discrete variables $X$ and $Y$, where $X$ has $m$
possible values and $Y$ has $n$ possible values.\\
\pause
\bigskip
We can create an $m \times n$ table of the frequencies of the
observations.\\
\pause
\bigskip
We can then conduct a chi-square test by comparing the observed frequencies to the expected
frequencies under the null hypothesis of independence.
\end{frame}

\begin{frame}
Let $O_{i}$ denote the observed frequency in position $i$ of the table
and $E_i$ denote the expected frequency in position $i$.\\
\pause
\bigskip
Then we have a test statistic $T$, where 
\pause
\begin{eqnarray*}
T &=& \sum_{i=1}^{m \times n} \frac{(O_i - E_i)^2}{E_i}\\
\end{eqnarray*}
\pause
$T$ is distributed $\chi^2$ with $(m-1) \times (n-1)$ degrees of freedom.\\
\bigskip
\pause
We calculate the test statistic and then see whether we can reject the
null hypothesis of independence.
\end{frame}

\begin{frame}
\frametitle{An Example}
Suppose we want to know whether the {\tt Parreg} and
{\tt Parcomp} variables in the POLITY dataset are independent (Treier
and Jackman, 2008).\\
\pause
\bigskip
We have the following observed frequencies in a $5 \times 6$ table
since {\tt Parreg} takes on 5 possible values and {\tt Parcomp} takes
on 6 possible values.
\pause
\footnotesize
\begin{table}
\begin{center}
\begin{tabular}{l|cccccc|c}
& \multicolumn{6}{c}{{\tt Parcomp}} & \\
\hline
{\tt Parreg} & 0 & 1 & 2 & 3 & 4 & 5 & Total\\
\hline
1 & 487 & 0 & 0 & 0 & 10 & 0 & 497 \\
2 & 96 & 0 & 0 & 740 & 583 & 0 & 1419\\
3 & 0 & 0 & 299 & 3509 & 76 & 0 & 3884\\
4 & 0 & 3878 & 1811 & 0 & 0 & 0 & 5689\\
5 & 0 & 0 & 0 & 0 & 116 & 2336 & 2452\\
\hline
Total & 583 & 3878 & 2110 & 4249 & 785 & 2336 & 13941\\
\end{tabular}
\end{center}
\end{table}
\normalsize
\end{frame}

\begin{frame}
\frametitle{Expected Frequencies}
\pause
We have the observed frequencies, but how do get the expected
frequencies under independence? \\
\bigskip
\pause
We can estimate the expected frequencies from the data by doing the following:\\
\end{frame}

\begin{frame}
Suppose we want to calculate the expected frequency for the top-left
position in the table where \red{{\tt Parcomp} $=0$ and {\tt Parreg} $=1$}. 
\footnotesize
\pause
\begin{table}
\begin{center}
\begin{tabular}{l|cccccc|c}
& \multicolumn{6}{c}{{\tt Parcomp}} & \\
\hline
{\tt Parreg} & 0 & 1 & 2 & 3 & 4 & 5 & Total\\
\hline
1 & \textcolor{red}{$E_1$} & $E_2$ & $E_3$ & $E_4$ &  $E_5$ & $E_6$ & \textbf{497} \\
2 & $E_7$ & $E_8$ & $E_9$ & $E_{10}$ &  $E_{11}$ & $E_{12}$ & 1419\\
3 &  $E_{13}$ & $E_{14}$ & $E_{15}$ & $E_{16}$ &  $E_{17}$ & $E_{18}$ & 3884\\
4 & $E_{19}$ & $E_{20}$ & $E_{21}$ & $E_{22}$ &  $E_{23}$ & $E_{24}$ & 5689\\
5 & $E_{25}$ & $E_{26}$ & $E_{27}$ & $E_{28}$ &  $E_{29}$ & $E_{30}$ & 2452\\
\hline
Total & \textbf{583} & 3878 & 2110 & 4249 & 785 & 2336 & \textbf{13941}\\
\end{tabular}
\end{center}
\end{table}
\normalsize
\pause
\begin{enumerate}
\item First calculate the proportion of the data where {\tt Parreg} $= 1$: \pause
$\frac{497}{13941} \approx 0.036$
\pause
\item Then find the expected frequency by taking that proportion and
multiplying it by the number of observations with {\tt Parcomp} $=0$:
\pause \red{$E_1 \approx 0.036 \times 583 \approx \mathbf{21}$} 
\pause
\end{enumerate}
\bigskip
We can also switch the order of {\tt Parreg} $=1$ and {\tt Parcomp} $=0$.
\end{frame}

\begin{frame}
We can then fill out the rest of the expected frequencies table using
the same method.
\pause
\footnotesize
\begin{table}
\begin{center}
\begin{tabular}{l|cccccc|c}
& \multicolumn{6}{c}{{\tt Parcomp}} & \\
\hline
{\tt Parreg} & 0 & 1 & 2 & 3 & 4 & 5 & Total\\
\hline
1 & 21 & 138 & 75 & 152 & 28 & 83 & \textbf{497} \\
2 & 59 & 395 & 215 & 432 & 80 & 238 & 1419\\
3 & 162 & 1080 & 588 & 1184 & 219 & 651 & 3884\\
4 & 238 & 1583 & 861 & 1734 & 320 & 953 & 5689\\
5 & 103 & 682 & 371 & 747 & 138 & 411 & 2452\\
\hline
Total & \textbf{583} & 3878 & 2110 & 4249 & 785 & 2336 & \textbf{13941}\\
\end{tabular}
\end{center}
\end{table}
\normalsize
\end{frame}

\begin{frame}[fragile]
We then calculate our test statistic $T$:
\pause
\tiny
\begin{eqnarray*}
T &=& \frac{(487-21)^2}{21} + \frac{(0-138)^2}{138} +
\frac{(0-75)^2}{75} + \frac{(0-152)^2}{152} + \frac{(10-28)^2}{28} +
\frac{(0-83)^2}{83} + \\
&\phantom{=}& \frac{(96-59)^2}{59} + \frac{(0-395)^2}{395} +
\frac{(0-215)^2}{215} + \frac{(740-432)^2}{432} + \frac{(583-80)^2}{80} +
\frac{(0-238)^2}{238} + \\
&\phantom{=}& \frac{(0-162)^2}{162} + \frac{(0-1080)^2}{1080} +
\frac{(299-588)^2}{588} + \frac{(3509-1184)^2}{1184} + \frac{(76-219)^2}{219} +
\frac{(0-651)^2}{651} + \\
&\phantom{=}& \frac{(0-238)^2}{238} + \frac{(3878-1583)^2}{1583} +
\frac{(1811-861)^2}{861} + \frac{(0-1734)^2}{1734} + \frac{(0-320)^2}{320} +
\frac{(0-953)^2}{953} + \\
&\phantom{=}& \frac{(0-103)^2}{103} + \frac{(0-682)^2}{682} +
\frac{(0-371)^2}{371} + \frac{(0-747)^2}{747} + \frac{(116-138)^2}{138} +
\frac{(2336-411)^2}{411} \\
\pause
&=& \mathbf{40290.79}
\end{eqnarray*}
\normalsize
\end{frame}

\begin{frame}[fragile]
We can then test the null hypothesis of independence by seeing the
area to the right of our $T$ in a $\chi^2_{20}$ distribution.
\tiny
\pause
\medskip
\begin{Schunk}
\begin{Sinput}
> pchisq(40290.79, df = 20, lower.tail = F)
\end{Sinput}
\begin{Soutput}
[1] 0
\end{Soutput}
\end{Schunk}
\normalsize
\pause
\bigskip
We can conclude that the two variables are clearly not independent since
the probability of getting a $T$ that is as extreme as our $T$ given
independence is 0.
\end{frame}

\end{document}
