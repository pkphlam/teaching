\documentclass[handout]{beamer}


\usetheme{default}
\usepackage{subfigure}
\usepackage{amsmath}
\usepackage{Sweave}
\usepackage{graphicx}
\usepackage{color}
\usepackage{amsfonts}
\usepackage{amssymb}
\usepackage{multicol}
\usepackage[all]{xy}
\usepackage{bm}


\author{Patrick Lam}
\title{Multilevel Models}
\date{}
%\date{December 1, 2008}

\begin{document}

\newcommand{\red}{\textcolor{red}}
\newcommand{\blue}{\textcolor{blue}}
\newcommand{\purple}{\textcolor{purple}}

\frame{\titlepage}

\begin{frame}
\frametitle{Outline}
\tableofcontents
\end{frame}

\section{Introduction}

\begin{frame}
\frametitle{Outline}
\tableofcontents[currentsection]
\end{frame}

\begin{frame}
\frametitle{Multilevel Data}
\pause
Suppose we have multilevel data (perhaps cross-country survey data) in
which we have
\pause
\begin{itemize}
\item $n$ individuals indexed by $i$
\pause
\item $J$ groups indexed by $j$ where each individual $i$ belongs to
one group $j$
\end{itemize}
\pause
\bigskip
We may have more than two levels (individuals
within countries within regions) \pause or non-nested levels
(individuals within certain age groups but also within certain levels
of education).\\
\pause
\bigskip
Levels can be interpreted in different ways \pause (in TSCS data, we have
countries over time where each observation is an individual and observations are grouped by country or time).\\
\pause
\bigskip
We may also have both individual and group-level covariates.
\end{frame}

\begin{frame}[fragile]
\frametitle{Running Example}
\pause
As a running example, we will use a dataset on levels of
radon in houses in different counties.
\pause
\tiny
\medskip
\begin{Schunk}
\begin{Sinput}
> radon <- read.csv("radon.csv")
\end{Sinput}
\end{Schunk}
\normalsize 
\pause
\begin{itemize}
\item Dependent variable ({\tt log.radon}): the log of the measured
radon level in a house
\pause
\item Independent variable ({\tt floor}): dummy variable for whether
the measurement was taken on the first floor (1) or the basement (0)
\pause
\item The houses are grouped by {\tt county} (85 counties total).
\pause
\item Group-level covariate ({\tt u.full}): a measurement of soil
uranium at the county level
\end{itemize}
\pause
\bigskip
So our units are houses and the houses are grouped by county.
\end{frame}


\begin{frame}
How do we deal with the multilevel nature of the data?
\pause
\bigskip
\begin{enumerate}
\item We can assume the multilevel nature of the data is irrelevant
by pooling all the observations together as if the groupings did not
exist (complete pooling model).
\pause
\medskip
\item We can assume the multilevel nature of the data gives us
information by allowing the coefficients to be completely different for
each group and estimating them individually for each group (no pooling model).
\pause
\medskip
\item We can allow the coefficients for each group to be different but
drawn from a common distribution (multilevel model).
\end{enumerate}
\pause
\bigskip
Options 2 and 3 allow us to model unobserved heterogeneity not
captured by our covariates \pause (for example, there may be something
special about group A that is not captured by our covariates)
\end{frame}

\section{The Complete Pooling Model}

\begin{frame}
\frametitle{Outline}
\tableofcontents[currentsection]
\end{frame}

\begin{frame}[fragile]
\frametitle{The Complete Pooling Model}
Suppose we have the following linear model:
\pause
\begin{eqnarray*}
y_i = \alpha + \beta x_i + \epsilon_i
\end{eqnarray*}
\pause
The \textbf{complete pooling model} assumes that the groupings provide
no information by constraining the coefficients to be the same for all
groups \pause (we basically pool all the observations together into a single group).\\
\bigskip
\pause
This is just the normal regression model where $\alpha$ and $\beta$ do
not vary by $j$.\\
\medskip
\pause 
\tiny
\begin{Schunk}
\begin{Sinput}
> complete.pool <- lm(log.radon ~ floor, data = radon)
\end{Sinput}
\end{Schunk}
\normalsize
\bigskip
\pause
The complete pooling approach may be too restrictive in that it
ignores any variation in the coefficients between groups \pause (for
example, $\beta$ might vary over time or across countries.)
\end{frame}

\section{The No Pooling (fixed effects) Model}

\begin{frame}
\frametitle{Outline}
\tableofcontents[currentsection]
\end{frame}


\begin{frame}
\frametitle{The No Pooling Model}
In the \textbf{no pooling model}, one or more of the coefficients are
allowed to vary by group.  \\
\pause
\bigskip
For example, we can let the intercept differ by group:
\begin{eqnarray*}
y_i = \alpha_{j[i]} + \beta x_i + \epsilon_i
\end{eqnarray*} 
where the notation $\alpha_{j[i]}$ denotes the $\alpha$ corresponding
to the group $j$ that individual $i$ belongs to.\\
\pause
\bigskip
This model is commonly known as the \textbf{fixed effects model}.\\
\end{frame}

\begin{frame}
The varying intercepts attempt to model unobserved heterogeneity
between groups.\\
\pause
\bigskip
For example, in political science papers, you usually see models that
use things like ``country fixed effects''.\\
\pause
\bigskip
The assumption is that there might be something different about being
in a specific country that cannot be captured by the covariates.\\
\pause
\bigskip
Therefore, each country is giving its own intercept (fixed effect) to
allow for a different baseline level of $y$ (when all covariates equal
0).\\
\pause
\bigskip
The ``fixed effects'' terminology can be very confusing since it can
also refer to slightly different quantities (as we will see later).
\end{frame}


\begin{frame}
We can also let the intercept and slope(s) differ by group:
\begin{eqnarray*}
y_i = \alpha_{j[i]} + \beta_{j[i]} x_i + \epsilon_i
\end{eqnarray*} 
\pause
This model is less common.\\
\bigskip
\pause
It assumes that the relationship between $x$ and $y$ is different for
each group.\\
\bigskip
\pause
One should almost never have a varying slope without a varying
intercept unless there is a good theoretical reason to do so.
\end{frame}

\begin{frame}
\frametitle{Dummy Variables Regression}
A common way to estimate the fixed effects model is to
include $J-1$ dummy indicators for the $J$ groups into the regression model (with one group left out as the baseline category)\\ 
\pause
\bigskip
This is sometimes called \textit{dummy variables} regression.\\
\pause
\bigskip
Suppose our individuals belong to one of three groups.  \pause  Let
$g_j$ be a dummy variable that takes on a value of 1 if $i$ is in
group $j$ and 0 if not.
\end{frame}

\begin{frame}[fragile]
The varying intercept (fixed effects) model is
\begin{eqnarray*}
y_i = \alpha + \beta_1 x_i + \beta_2 g_2 + \beta_3 g_3 + \epsilon_i
\end{eqnarray*}
\pause
where 
\begin{eqnarray*}
\alpha_1 &=& \alpha \\
\pause
\alpha_2 &=& \alpha + \beta_2\\
\pause
\alpha_3 &=& \alpha + \beta_3\\ 
\end{eqnarray*}
\pause
$\beta_2$ and $\beta_3$ can be interpreted as the effect on $y$ of being in
group 2 and group 3 relative to group 1 (the baseline category).\\
\pause
\bigskip
In our example, we can add ``county fixed effects'' by including
county dummies in the regression.
\tiny
\medskip
\pause
\begin{Schunk}
\begin{Sinput}
> no.pool <- lm(log.radon ~ floor + as.factor(county), data = radon)
\end{Sinput}
\end{Schunk}
\normalsize
\end{frame}

\begin{frame}


\end{frame}

\subsection{The Hierarchical (random effects) Model}

\begin{frame}
\frametitle{Outline}
\tableofcontents[currentsubsection]
\end{frame}

\begin{frame}
\frametitle{The Hierarchical Model}
\pause
The \textbf{hierarchical model} is a compromise between the complete pooling
and no pooling models in that it partially pools the coefficients by
allowing them to differ according to a common distribution (usually
the normal distribution). \\
\bigskip
\pause
Hierarchical models are also known as partially pooled models, \pause
multi-level models, \pause mixed effects models, \pause random effects
models, or \pause random coefficient models.
\end{frame}

\begin{frame}
The varying intercept model can be written as 
\begin{eqnarray*}
y_i &=& \alpha_{j[i]} + \beta x_i + \epsilon_i \\
\alpha_j &\sim& N(\alpha, \sigma^2_{\alpha})
\end{eqnarray*}
\pause
or equivalently as
\begin{eqnarray*}
y_i &=& \alpha_{j[i]} + \beta x_i + \epsilon_i \\
\alpha_j &=& \alpha + \eta_j \\
\eta_j &\sim& N(0, \sigma^2_{\alpha})\\
\end{eqnarray*}
\pause
The estimate of the variance term $\sigma^2_{\alpha}$ gives us a
measure of how different the coefficients are between groups.
\pause
\begin{itemize}
\item The complete pooling model assumes $\sigma^2_{\alpha} = 0$ 
\pause
\item The no pooling model assumes $\sigma^2_{\alpha} = \infty$ 
\end{itemize}
\end{frame}

\begin{frame}
We can also incorporate group-level covariates into the model. \\
\pause
\bigskip
Suppose we think that some group-level covariate $u$ has an effect on
the average $y$ within groups.  \pause We can think of $u$ as
affecting the intercepts $\alpha_j$.
\pause
\begin{eqnarray*}
y_i &=& \alpha_{j[i]} + \beta x_i + \epsilon_i \\
\alpha_j &\sim& N(\gamma_0 + \gamma_1 u_j, \sigma^2_{\alpha})
\end{eqnarray*}
\pause
or equivalently 
\begin{eqnarray*}
y_i &=& \alpha_{j[i]} + \beta x_i + \epsilon_i \\
\alpha_j &=& \gamma_0 + \gamma_1 u_j + \eta_j \\
\eta_j &\sim& N(0, \sigma^2_{\alpha})
\end{eqnarray*}
\pause
We can interpret $\gamma_1$ as the effect of $u$ on the average $y$
within a group when all the individual-level covariates are 0.
\end{frame}

\begin{frame}
The varying intercept and slope(s) model can be written as 
\begin{eqnarray*}
y_i &=& \alpha_{j[i]} + \beta_{j[i]} x_i + \epsilon_i \\
\begin{pmatrix}
\alpha_j\\
\beta_j\\
\end{pmatrix} 
&\sim& N \left( 
\begin{pmatrix}
\alpha \\
\beta
\end{pmatrix},
\begin{pmatrix}
\sigma^2_{\alpha} & \rho \sigma_{\alpha} \sigma_{\beta} \\
\rho \sigma_{\alpha} \sigma_{\beta} & \sigma^2_{\beta} \\
\end{pmatrix}
\right)\\
\end{eqnarray*}
\pause
or with group-level covariates
\begin{eqnarray*}
y_i &=& \alpha_{j[i]} + \beta_{j[i]} x_i + \epsilon_i \\
\begin{pmatrix}
\alpha_j\\
\beta_j\\
\end{pmatrix} 
&\sim& N \left( 
\begin{pmatrix}
\gamma_0^{\alpha} + \gamma_1^{\alpha} u_j \\
\gamma_0^{\beta} + \gamma_1^{\beta} u_j
\end{pmatrix},
\begin{pmatrix}
\sigma^2_{\alpha} & \rho \sigma_{\alpha} \sigma_{\beta} \\
\rho \sigma_{\alpha} \sigma_{\beta} & \sigma^2_{\beta} \\
\end{pmatrix}
\right)\\
\end{eqnarray*}
\pause
We can interpret $\gamma_1^{\beta}$ as the effect of $u$ on the
relationship between $x$ and $y$ within a group.
\end{frame}

\begin{frame}
Hierarchical models are often referred to as \textbf{random effects
models} because of the randomness of the distribution of coefficients.\\
\bigskip
\pause
However, they are also commonly called \textbf{mixed effects} models
because they consist of two types of effects:
\pause
\begin{itemize}
\item \textit{fixed effects}, which do not vary across groups
\pause
\item \textit{random effects}, which do vary across groups
\end{itemize}
\pause
\bigskip
The terminology is confusing because the fixed effects referred to
here are NOT the same as the fixed effects in the no pooling model,
which refer to the varying intercepts.
\end{frame}

\begin{frame}
In the simple varying intercepts hierarchical model, we can identify
the \blue{fixed effects} and \red{random effects} components.
\pause
\begin{eqnarray*}
y_i &=& \red{\alpha_{j[i]}} + \blue{\beta} x_i + \epsilon_i \\
%\red{\alpha_j} &\sim& N(\blue{\gamma_0} + \blue{\gamma_1} u_j, \sigma^2_{\alph%a})
\red{\alpha_j} &=& \blue{\gamma_0} + \blue{\gamma_1} u_j + \red{\eta_j}
\end{eqnarray*}
\pause
By contrast, in the no pooling model, the ``fixed effects'' referred
to $\alpha_j$, which was allowed to vary.\\
\bigskip
\pause
The confusion in terminology have led some (Gelman, among others) to
call for the end to the use of the terms ``fixed effects'' and ``random
effects''.\\
\bigskip
\pause
Nevertheless, people still use them.\\
\bigskip
\pause
In substantive papers, the mention of ``fixed effects'' almost always refers to the no
pooling model with intercepts varying by group.
\end{frame}

\subsection{Fixed Effects versus Random Effects}

\begin{frame}
\frametitle{Outline}
\tableofcontents[currentsubsection]
\end{frame}

\begin{frame}
\frametitle{Fixed Effects versus Random Effects}
\pause
Unless there is a theoretical expectation that a complete pooling
model is appropriate (when there is nothing important about the
groupings), one should always take advantage of the information in the
hierarchical nature of the data.\\
\pause
\bigskip
However, which model is better for dealing with hierarchical data: the
fixed effects (no pooling) model or the random effects
(partial pooling/hierarchical/mixed effects) model?
\end{frame}

\begin{frame}
Many people have suggested reasons for using one over the other, such as
\pause
\begin{itemize}
\item use fixed effects if the group-level coefficients are of
interest, and use random effects if the interest lies in the
underlying population
\pause
\item use fixed effects when the groups in the data represent all
possible groups, and use random effects when the population includes
groups not in the data
\pause
\item use fixed effects when the number of groups is small
(less than five) since there may not be enough to accurately estimate
the variation in the coefficients
\end{itemize}
\bigskip
\pause
Gelman suggests always using the random effects model because it is
less restrictive (the variance of the coefficients is estimated rather
than assumed to be infinite).
\end{frame}

\section{Classical Approach to Hierarchical Models in R}

\begin{frame}
\frametitle{Outline}
\tableofcontents[currentsection]
\end{frame}




\subsection{The Complete Pooling Model}

\begin{frame}
\frametitle{Outline}
\tableofcontents[currentsubsection]
\end{frame}

\begin{frame}[fragile]
\frametitle{The Complete Pooling Model}
\pause
The complete pooling model is the simple linear regression model, which
ignores the groupings:
\pause
\begin{eqnarray*}
y_i = \alpha + \beta \cdot floor + \epsilon_i
\end{eqnarray*}
\tiny
\pause
\begin{Schunk}
\begin{Sinput}
> complete.pool <- lm(log.radon ~ floor, data = radon)
> complete.pool
\end{Sinput}
\begin{Soutput}
Call:
lm(formula = log.radon ~ floor, data = radon)

Coefficients:
(Intercept)        floor  
      1.327       -0.613  
\end{Soutput}
\end{Schunk}
\normalsize
\end{frame}

\subsection{The No Pooling (fixed effects) Model}

\begin{frame}
\frametitle{Outline}
\tableofcontents[currentsubsection]
\end{frame}

\begin{frame}[fragile]
\frametitle{The No Pooling Model}
\pause
The no pooling (fixed effects) model adds dummy indicator variables
for the groupings (``county fixed effects'').

\bigskip
\pause
We have 86 coefficients corresponding to the intercept (baseline
county), slope, and $m-1$ (84) county dummies.
\tiny
\medskip
\pause
\begin{Schunk}
\begin{Sinput}
> length(coef(no.pool))
\end{Sinput}
\begin{Soutput}
[1] 86
\end{Soutput}
\end{Schunk}
\normalsize
\pause
\bigskip
Alternatively, we could have estimated this without an intercept term
but with $m$ county dummies.
\tiny
\medskip
\pause
\begin{Schunk}
\begin{Sinput}
> no.pool.1 <- lm(log.radon ~ floor - 1 + as.factor(county), data = radon)
> length(coef(no.pool.1))
\end{Sinput}
\begin{Soutput}
[1] 86
\end{Soutput}
\end{Schunk}
\normalsize
\end{frame}

\subsection{The Hierarchical (random effects) Model}

\begin{frame}
\frametitle{Outline}
\tableofcontents[currentsubsection]
\end{frame}

\begin{frame}
\frametitle{The Hierarchical Model}
\pause
Remember that in our hierarchical/mixed effects model, we had two
components: \blue{fixed effects} and \red{random effects}.\\
\bigskip
\pause
Our varying intercept model is
\begin{eqnarray*}
y_i &=& \red{\alpha_{j[i]}} + \blue{\beta} x_i + \epsilon_i \\
\red{\alpha_j} &=& \blue{\gamma_0} + \blue{\gamma_1} u_j + \red{\eta_j}
\end{eqnarray*}
\pause
We can substitute the second equation into the first to get
\begin{eqnarray*}
y_i &=& (\blue{\gamma_0} + \red{\eta_{j[i]}}) + \blue{\beta} x_i  +
\blue{\gamma_1} u_{j[i]} + \epsilon_i
\end{eqnarray*}
\end{frame}

\end{document}
