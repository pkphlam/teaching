\documentclass[handout]{beamer}


\usetheme{default}
\usepackage{subfigure}
\usepackage{amsmath}
\usepackage{Sweave}
\usepackage{graphicx}
\usepackage{color}
\usepackage{multicol}
\usepackage{bm}


\author{Patrick Lam}
\title{Conjugate Models}
\date{}
%\date{September 29, 2008}

\begin{document}

\newcommand{\red}{\textcolor{red}}
\newcommand{\blue}{\textcolor{blue}}
\newcommand{\purple}{\textcolor{purple}}

\frame{\titlepage}

\begin{frame}
\frametitle{Outline}
\tableofcontents
\end{frame}

\section{Conjugate Models}

\begin{frame}
\frametitle{Outline}
\tableofcontents[currentsection]
\end{frame}

\subsection{What is Conjugacy?}

\begin{frame}
\frametitle{Outline}
\tableofcontents[currentsubsection]
\end{frame}

\begin{frame}
\frametitle{Conjugacy}
\pause
Suppose we have a Bayesian model with a likelihood $p(y | \theta)$ and
a \red{prior $p(\theta)$}.\\
\pause
\bigskip
If we multiply our likelihood and \red{prior}, we get our \blue{posterior
$p(\theta | y)$} up to a constant of proportionality.\\
\pause
\bigskip
If our \blue{posterior} is a distribution that is of the same family as our
\red{prior}, then we have \textit{conjugacy}.  \pause We say that the
\red{prior} is conjugate to the likelihood.\\
\pause
\bigskip
Conjugate models are great because we know the exact distribution of
the \blue{posterior} so we can easily simulate or derive quantities of
interest analytically.\\
\pause
\bigskip
In practice, we rarely have conjugacy.
\end{frame}

\begin{frame}
\frametitle{Brief List of Conjugate Models}
\pause
\scriptsize
\begin{table}
\begin{center}
\begin{tabular}{c|c|c}
Likelihood & Prior & Posterior\\
\hline
Binomial & Beta & Beta\\
Negative Binomial & Beta & Beta\\
Poisson & Gamma & Gamma\\
Geometric & Beta & Beta\\
Exponential & Gamma & Gamma\\
Normal (mean unknown) & Normal & Normal\\
Normal (variance unknown) & Inverse Gamma & Inverse Gamma\\
Normal (mean and variance unknown) & Normal/Gamma & Normal/Gamma\\
Multinomial & Dirichlet & Dirichlet
\end{tabular}
\end{center}
\end{table}
\normalsize
\end{frame}

\subsection{The Beta-Binomial Model}

\begin{frame}
\frametitle{Outline}
\tableofcontents[currentsubsection]
\end{frame}


\begin{frame}
\frametitle{A Binomial Example}
\pause
Suppose we have vector of data on voter turnout for a random sample of $n$
voters in the 2004 US Presidential election.   \\
\pause
\bigskip
We can model the voter turnout with a binomial model.\\
\pause
\begin{equation*}
Y \sim \mathrm{Binomial}(n, \pi)
\end{equation*}
\pause
\bigskip
Quantity of interest: $\pi$ (voter turnout)\\
\pause
\bigskip
Assumptions:
\pause
\begin{itemize}
\item Each voter's decision to vote follows the Bernoulli
distribution. \pause
\item Each voter has the same probability of voting.  \pause
(unrealistic) \pause
\item Each voter's decision to vote is independent.  \pause (unrealistic)
\end{itemize}
\end{frame}

%\subsection{The Conjugate Beta Prior}

\begin{frame}
\frametitle{The Conjugate Beta Prior}
\pause
We can use the beta distribution as a \textcolor{red}{prior} for $\pi$, since the beta
distribution is conjugate to the binomial distribution.\\
\pause
\begin{eqnarray*}
\textcolor{blue}{p(\pi | y)} &\propto& p(y | \pi) \textcolor{red}{p(\pi)} \\
\pause
&=& \mathrm{Binomial}(n, \pi) \times
\textcolor{red}{\mathrm{Beta}(\alpha, \beta)}\\
\pause
&=& \binom{n}{y} \pi^y (1 - \pi)^{(n-y)} \textcolor{red}{\frac{\Gamma (\alpha + \beta)}{\Gamma (\alpha)
\Gamma (\beta)} \pi^{(\alpha - 1)} (1 - \pi)^{(\beta-1)}}\\
\pause
&\propto& \pi^y (1 - \pi)^{(n-y)} \textcolor{red}{\pi^{(\alpha - 1)}
(1 - \pi)^{(\beta-1)}} \\\\
\pause
\textcolor{blue}{p(\pi | y)} &\textcolor{blue}{\propto}& \textcolor{blue}{\pi^{y+\alpha-1} (1-\pi)^{n-y+\beta-1}}
\end{eqnarray*}
\pause
The \textcolor{blue}{posterior distribution} is simply a
\textcolor{blue}{Beta($y+\alpha, n-y+\beta$)} distribution. \pause
Effectively, our \textcolor{red}{prior} is just adding $\alpha-1$
successes and $\beta - 1$ failures to the dataset.
\end{frame}

\begin{frame}
\frametitle{The Uninformative (Flat) Uniform Prior}
\pause
Suppose we have no strong prior beliefs about the parameters.  \pause
We can choose a \textcolor{red}{prior} that gives equal weight to all possible values
of the parameters, essentially an uninformative or ``flat'' \textcolor{red}{prior}. \pause
\begin{equation*}
\textcolor{red}{p(\pi) = \mathrm{constant}}
\end{equation*}
for all values of $\pi$. \\
\pause
\bigskip
For the binomial model, one example of a flat \textcolor{red}{prior} is the \textcolor{red}{Beta(1,1) prior}:
\pause
\begin{eqnarray*}
\textcolor{red}{p(\pi)} &\textcolor{red}{=}& \textcolor{red}{\frac{\Gamma (2)}{\Gamma (1)
\Gamma (1)} \pi^{(1 - 1)} (1 - \pi)^{(1-1)}} \\
\pause
&\textcolor{red}{=}& \textcolor{red}{1}
\end{eqnarray*}
\pause
which is the Uniform distribution over the $[0,1]$ interval.
\end{frame}

\begin{frame}[fragile]
Since we know that a Binomial likelihood and a
\textcolor{red}{Beta(1,1) prior} produces a \textcolor{blue}{Beta($y+1,
n-y+1$) posterior}, we can simulate the \textcolor{blue}{posterior} in R.\\
\pause
\bigskip
Suppose our turnout data had 500 voters, of which 285 voted.
\pause
\medskip
\tiny{

\begin{Schunk}
\begin{Sinput}
> table(turnout)
\end{Sinput}
\begin{Soutput}
turnout
  0   1 
215 285 
\end{Soutput}
\end{Schunk}
} 
\pause
\bigskip
\normalsize{
Setting our \textcolor{red}{prior} parameters at $\alpha = 1$ and $\beta = 1$,
}
\medskip
\pause
\tiny{
\begin{Schunk}
\begin{Sinput}
> a <- 1
> b <- 1
\end{Sinput}
\end{Schunk}
} 
\pause
\bigskip
\normalsize{
we get the \textcolor{blue}{posterior}
}
\medskip
\pause
\tiny{
\begin{Schunk}
\begin{Sinput}
> posterior.unif.prior <- rbeta(10000, shape1 = 285 + a, shape2 = 500 - 
+     285 + b)
\end{Sinput}
\end{Schunk}
}
\normalsize
\end{frame}

\section{The Normal Model}

\begin{frame}
\frametitle{Outline}
\tableofcontents[currentsection]
\end{frame}

\subsection{Normal Model with Unknown Mean, Known Variance}

\begin{frame}
\frametitle{Outline}
\tableofcontents[currentsubsection]
\end{frame}

\begin{frame}
\frametitle{Normal Model with Unknown Mean, Known Variance}
\pause
Suppose we wish to estimate a model where the likelihood of the data
is normal with an unknown mean $\mu$ and a known variance $\sigma^2$.\\
\bigskip
\pause
Our parameter of interest is $\mu$. \\
\bigskip
\pause
We can use a conjugate \red{Normal prior} on $\mu$, with mean \red{$\mu_0$} and
variance \red{$\tau^2_0$}.
\bigskip
\pause
\begin{eqnarray*}
\blue{p(\mu | \bm{y}, \sigma^2)} &\propto& p(\bm{y} | \mu, \sigma^2)
\red{p(\mu)} \\
\pause
\blue{\mathrm{Normal}(\mu_1, \tau^2_1)} &=& \mathrm{Normal}(\mu, \sigma^2)
\times \red{\mathrm{Normal}(\mu_0, \tau^2_0)}
\end{eqnarray*}
\end{frame}

\begin{frame}
Let $\theta$ represent our parameter of interest, in this case $\mu$.
\pause
\footnotesize
\begin{eqnarray*}
\blue{p(\theta | \bm{y})} &\propto& \prod_{i=1}^n \frac{1}{\sqrt{2 \pi
\sigma^2}} \exp \left( -\frac{(y_i-\theta)^2}{2\sigma^2} \right) \times
\red{\frac{1}{\sqrt{2 \pi \tau_0^2}} \exp \left( -\frac{(\theta - \mu_0
)^2}{2 \tau_0^2} \right)} \\
\pause
&\propto& \exp \left( -\sum_{i=1}^n \frac{(y_i-\theta)^2}{2\sigma^2}
\red{-\frac{(\theta - \mu_0)^2}{2 \tau_0^2}} \right) \\
\pause
&=& \exp \left[ \purple{-\frac{1}{2}} \left( \sum_{i=1}^n
\frac{(y_i-\theta)^2}{\sigma^2} \red{+\frac{(\theta - \mu_0)^2}{\tau_0^2}} \right) \right] \\
\pause
&=& \exp \left[ \purple{-\frac{1}{2 \sigma^2 \tau_0^2}} \left(
\purple{\tau_0^2} \sum_{i=1}^n (y_i-\theta)^2 \red{+}\purple{\sigma^2}\red{(\theta - \mu_0)^2} \right) \right] \\
\pause
&=& \exp \left[ \purple{-\frac{1}{2 \sigma^2 \tau_0^2}} \left(
\purple{\tau_0^2} \sum_{i=1}^n (y_i^2 - 2\theta y_i + \theta^2)
\red{+}\purple{\sigma^2}\red{(\theta^2 - 2 \theta \mu_0 + \mu_0^2)} \right) \right] \\
\end{eqnarray*}
\normalsize
\end{frame}

\begin{frame}
We can multiply the $2\theta y_i$ term in the summation by
\textcolor{cyan}{$\frac{n}{n}$} in order to get the equations in
terms of the sufficient statistic $\bar{y}$.
\pause
\footnotesize
\begin{eqnarray*}
\blue{p(\theta | \bm{y})} &\propto& \exp \left[ \purple{-\frac{1}{2 \sigma^2
\tau_0^2}} \left( \purple{\tau_0^2} \sum_{i=1}^n (y_i^2 - 2\theta
\textcolor{cyan}{\frac{n}{n}} y_i + \theta^2) \red{+}\purple{\sigma^2}\red{(\theta^2 - 2 \theta \mu_0 + \mu_0^2)} \right) \right] \\
\pause
&=& \exp \left[ \purple{-\frac{1}{2 \sigma^2
\tau_0^2}} \left( \purple{\tau_0^2} \sum_{i=1}^n y_i^2 -
\purple{\tau_0^2} 2\theta n \bar{y} + \purple{\tau_0^2} n \theta^2 \red{+\theta^2}\purple{\sigma^2} \red{- 2 \theta \mu_0}
\purple{\sigma^2} + \red{\mu_0^2} \purple{\sigma^2} \right) \right] \\
\pause
\end{eqnarray*}
\normalsize
We can then factor the terms into several parts.  Since $\red{\mu_0^2}
\purple{\sigma^2}$ and $\purple{\tau_0^2} \sum_{i=1}^n y_i^2$ do not
contain $\theta$, we can represent them as some constant
\textcolor{brown}{$k$}, which we will drop into the normalizing constant.
\pause
\footnotesize
\begin{eqnarray*}
\blue{p(\theta | \bm{y})} &\propto& \exp \left[ \purple{-\frac{1}{2 \sigma^2
\tau_0^2}} \left( \purple{\theta^2}\left(\purple{\sigma^2} +
\purple{\tau_0^2} n \right) - \purple{2 \theta} \left( \red{\mu_0}
\purple{\sigma^2} + \purple{\tau_0^2} n \bar{y} \right) + \textcolor{brown}{k} \right) \right] \\
\pause
&=& \exp \left[ \purple{-\frac{1}{2}} \left( \purple{\theta^2}\left(\frac{\purple{\sigma^2} +
\purple{\tau_0^2} n}{\purple{\sigma^2 \tau_0^2}} \right) - \purple{2
\theta} \left( \frac{\red{\mu_0} \purple{\sigma^2} + \purple{\tau_0^2}
n \bar{y}}{\purple{\sigma^2 \tau_0^2}} \right)  + \textcolor{brown}{k} \right) \right] \\
\pause
&=& \exp \left[ \purple{-\frac{1}{2}} \left(
\purple{\theta^2} \left( \frac{\purple{1}}{\purple{\tau_0^2}} +
\frac{n}{\purple{\sigma^2}} \right) - \purple{2 \theta} \left(
\frac{\red{\mu_0}}{\purple{\tau_0^2}} + \frac{n
\bar{y}}{\purple{\sigma^2}} \right)  + \textcolor{brown}{k} \right) \right] 
\end{eqnarray*}
\normalsize
\end{frame}

\begin{frame}
Let's multiply by $\frac{ \left(
\frac{1}{\tau_0^2} + \frac{n}{\sigma^2} \right)}{ \left( \frac{1}{\tau_0^2} +
\frac{n}{\sigma^2} \right)}$ in order to simplify the $\theta^2$ term.
\pause
\footnotesize
\begin{eqnarray*}
\blue{p(\theta | \bm{y})} &\propto& \exp \left[ \purple{-\frac{1}{2}}
\left( \frac{1}{\tau_0^2} +
\frac{n}{\sigma^2} \right)\left(
\purple{\theta^2} \left( \frac{\frac{\purple{1}}{\purple{\tau_0^2}} +
\frac{n}{\purple{\sigma^2}}}{\frac{1}{\tau_0^2} +
\frac{n}{\sigma^2}} \right) - \purple{2 \theta} \left(
\frac{\frac{\red{\mu_0}}{\purple{\tau_0^2}} + \frac{n
\bar{y}}{\purple{\sigma^2}}}{\frac{1}{\tau_0^2} +
\frac{n}{\sigma^2}} \right)  + \textcolor{brown}{k} \right) \right] \\
\pause
&=&  \exp \left[ \purple{-\frac{1}{2}}
\left( \frac{1}{\tau_0^2} +
\frac{n}{\sigma^2} \right)\left(
\purple{\theta^2}  - \purple{2 \theta} \left(
\frac{\frac{\red{\mu_0}}{\purple{\tau_0^2}} + \frac{n
\bar{y}}{\purple{\sigma^2}}}{\frac{1}{\tau_0^2} +
\frac{n}{\sigma^2}} \right)  + \textcolor{brown}{k} \right) \right] \\
\pause
&=& \exp \left[ \purple{-\frac{1}{2}}
\left( \frac{1}{\tau_0^2} +
\frac{n}{\sigma^2} \right)\left(
\purple{\theta}  - \left(
\frac{\frac{\red{\mu_0}}{\purple{\tau_0^2}} + \frac{n
\bar{y}}{\purple{\sigma^2}}}{\frac{1}{\tau_0^2} +
\frac{n}{\sigma^2}} \right) \right)^2 \right] \\
\end{eqnarray*}
\pause
\normalsize
Finally, we have something that looks like the density function of a
Normal distribution!
\end{frame}

\begin{frame}
\footnotesize
\begin{eqnarray*}
\blue{p(\theta | \bm{y})} &\propto&  \exp \left[ -\frac{1}{2}
\left( \frac{1}{\tau_0^2} +
\frac{n}{\sigma^2} \right)\left(
\theta  - \left(
\frac{\frac{\mu_0}{\tau_0^2} + \frac{n
\bar{y}}{\sigma^2}}{\frac{1}{\tau_0^2} +
\frac{n}{\sigma^2}} \right) \right)^2 \right] \\
\end{eqnarray*}
\pause
\normalsize
Posterior Mean: \blue{$\mu_1$} = $\frac{\left( \frac{\mu_0}{\tau_0^2} + \frac{n
\bar{y}}{\sigma^2}\right)}{\left( \frac{1}{\tau_0^2} +
\frac{n}{\sigma^2}\right)}$ \\
\bigskip
\pause
Posterior Variance: \blue{$\tau_1^2$} = $\left( \frac{1}{\tau_0^2} +
\frac{n}{\sigma^2} \right)^{-1}$\\
\bigskip
\pause
Posterior Precision: \blue{$\frac{1}{\tau_1^2}$} =
$\red{\frac{1}{\tau_0^2}} + \frac{n}{\sigma^2} $\\
\bigskip
\pause
\blue{Posterior Precision} is just the sum of the \red{prior
precision} and the data precision.
\end{frame}

\begin{frame}
We can also look more closely at how the prior mean $\red{\mu_0}$ and
the posterior mean $\blue{\mu_1}$ relate to each other.
\pause
\footnotesize
\begin{eqnarray*}
 \blue{\mu_1} &=& \frac{\left( \frac{\red{\mu_0}}{\tau_0^2} + \frac{n
\bar{y}}{\sigma^2}\right)}{\left( \frac{1}{\tau_0^2} +
\frac{n}{\sigma^2}\right)}\\
\pause
&=& \frac{\frac{\red{\mu_0} \sigma^2 + \tau_0^2 n \bar{y}}{\tau_0^2
\sigma^2}}{\frac{\sigma^2 + n\tau_0^2}{\tau_0^2 \sigma^2}} \\
\pause
&=& \frac{\red{\mu_0} \sigma^2 + \tau^2_0 n \bar{y}}{\sigma^2 + n\tau_0^2}\\
\pause
&=& \frac{\red{\mu_0} \sigma^2}{\sigma^2 + n\tau_0^2} + \frac{\tau^2_0 n \bar{y}}{\sigma^2 + n\tau_0^2}
\end{eqnarray*}
\normalsize
\pause
\begin{itemize}
\item As $n$ increases, data mean dominates prior mean.
\pause
\item As $\tau_0^2$ decreases (less prior variance, greater prior
precision), our prior mean becomes more important.
\end{itemize}
\end{frame}


\begin{frame}[fragile]
\frametitle{A Simple Example}
\pause
Suppose we have some (fake) data on the heights (in inches) of a
random sample of 100 individuals in the U.S. population.
\pause
\medskip
\tiny
\begin{Schunk}
\begin{Sinput}
> known.sigma.sq <- 16
> unknown.mean <- 68
> n <- 100
> heights <- rnorm(n, mean = unknown.mean, sd = sqrt(known.sigma.sq))
\end{Sinput}
\end{Schunk}
\normalsize
\pause
\bigskip
 We believe that the heights are normally distributed with some
unknown mean $\mu$ and a known variance $\sigma^2 = 16$.\\
\bigskip
\pause
Suppose before we see the data, we have a prior belief about the
distribution of $\mu$.  Let our prior mean \red{$\mu_0 = 72$} and our prior
variance \red{$\tau_0^2 = 36$.}
\medskip
\pause
\tiny
\begin{Schunk}
\begin{Sinput}
> mu0 <- 72
> tau.sq0 <- 36
\end{Sinput}
\end{Schunk}
\end{frame}

\begin{frame}[fragile]
Our posterior is a Normal distribution with Mean $\frac{\left( \frac{\mu_0}{\tau_0^2} + \frac{n
\bar{y}}{\sigma^2}\right)}{\left( \frac{1}{\tau_0^2} +
\frac{n}{\sigma^2}\right)}$ and Variance $\left( \frac{1}{\tau_0^2} +
\frac{n}{\sigma^2} \right)^{-1}$
\medskip
\pause
\tiny
\begin{Schunk}
\begin{Sinput}
> post.mean <- (mu0/tau.sq0 + (n * mean(heights)/known.sigma.sq))/(1/tau.sq0 + 
+     n/known.sigma.sq)
> post.mean
\end{Sinput}
\begin{Soutput}
[1] 68.03969
\end{Soutput}
\begin{Sinput}
> post.var <- 1/(1/tau.sq0 + n/known.sigma.sq)
> post.var
\end{Sinput}
\begin{Soutput}
[1] 0.1592920
\end{Soutput}
\end{Schunk}
\normalsize
\end{frame}



\subsection{Normal Model with Known Mean, Unknown Variance}

\begin{frame}
\frametitle{Outline}
\tableofcontents[currentsubsection]
\end{frame}

\begin{frame}
\frametitle{Normal Model with Known Mean, Unknown Variance}
\pause
Now suppose we wish to estimate a model where the likelihood of the data
is normal with a known mean $\mu$ and an unknown variance $\sigma^2$.\\
\bigskip
\pause
Now our parameter of interest is $\sigma^2$. \\
\bigskip
\pause
We can use a conjugate \red{inverse gamma prior} on $\sigma^2$, with
shape parameter \red{$\alpha_0$} and scale parameter \red{$\beta_0$}.
\bigskip
\pause
\begin{eqnarray*}
\blue{p(\sigma^2 | \bm{y}, \mu)} &\propto& p(\bm{y} | \mu, \sigma^2)
\red{p(\sigma^2)} \\
\pause
\blue{\mathrm{Invgamma}(\alpha_1, \beta_1)} &=& \mathrm{Normal}(\mu, \sigma^2)
\times \red{\mathrm{Invgamma}(\alpha_0, \beta_0)}
\end{eqnarray*}
\end{frame}

\begin{frame}
Let $\theta$ represent our parameter of interest, in this case $\sigma^2$.
\pause
\footnotesize
\begin{eqnarray*}
\blue{p(\theta | \bm{y}, \mu)} &\propto& \prod^n_{i=1}
\frac{1}{\sqrt{2\pi \theta}} \exp \left( -\frac{(y_i-\mu)^2}{2\theta}
\right) \times \red{\frac{\beta_0^{\alpha_0}}{\Gamma(\alpha_0)}
\theta^{-(\alpha_0+1)} \exp \left( - \frac{\beta_0}{\theta} \right)}\\
\pause
&\propto& \prod^n_{i=1} \theta^{-\frac{1}{2}} \exp \left(
-\frac{(y_i-\mu)^2}{2\theta} \right) \times
\red{\theta^{-(\alpha_0+1)} \exp \left( - \frac{\beta_0}{\theta} \right)}\\
\pause
&=& \theta^{-\frac{n}{2}} \exp \left(
-\frac{\sum_{i=1}^n (y_i-\mu)^2}{2\theta} \right) \times
\red{\theta^{-(\alpha_0+1)} \exp \left( - \frac{\beta_0}{\theta} \right)}\\
\pause
&=& \theta^{-(\red{\alpha_0} + \frac{n}{2} \red{+ 1})} \exp \left[ - \left( \red{\frac{\beta_0}{\theta}}
+\frac{\sum_{i=1}^n (y_i-\mu)^2}{2\theta} \right) \right] \\
\pause
&=& \theta^{-(\red{\alpha_0} + \frac{n}{2} \red{+ 1})}
\exp \left[ - \left( \frac{\red{2\beta_0}
+2\left( \frac{\sum_{i=1}^n (y_i-\mu)^2}{2}\right)}{\purple{2\theta}} \right) \right] \\
\pause
&=& \theta^{-(\red{\alpha_0} + \frac{n}{2} \red{+ 1})}
\exp \left[ - \left( \frac{\red{\beta_0}
+ \frac{\sum_{i=1}^n (y_i-\mu)^2}{2}}{\purple{\theta}} \right) \right] \\
\end{eqnarray*} 
\normalsize
\pause
This looks like the density of an inverse gamma distribution!
\end{frame}

\begin{frame}
\footnotesize
\begin{eqnarray*}
\blue{p(\theta | \bm{y}, \mu)} &\propto& \theta^{-(\alpha_0 +
\frac{n}{2} + 1)} \exp \left[ - \left( \frac{\beta_0
+ \frac{\sum_{i=1}^n (y_i-\mu)^2}{2}}{\theta} \right) \right] 
\end{eqnarray*}
\pause
\begin{eqnarray*}
\blue{\alpha_1} &=& \alpha_0 + \frac{n}{2}\\
\pause
\blue{\beta_1} &=& \beta_0 + \frac{\sum_{i=1}^n (y_i-\mu)^2}{2}
\end{eqnarray*}
\normalsize
\pause
Our posterior is an \blue{Invgamma($\alpha_0 + \frac{n}{2},\beta_0 + \frac{\sum_{i=1}^n (y_i-\mu)^2}{2}$)} distribution.
\end{frame}

\begin{frame}[fragile]
\frametitle{A Simple Example}
\pause
Again suppose we have some (fake) data on the heights (in inches) of a
random sample of 100 individuals in the U.S. population.
\pause
\medskip
\tiny
\begin{Schunk}
\begin{Sinput}
> known.mean <- 68
> unknown.sigma.sq <- 16
> n <- 100
> heights <- rnorm(n, mean = known.mean, sd = sqrt(unknown.sigma.sq))
\end{Sinput}
\end{Schunk}
\normalsize
\pause
\bigskip
 We believe that the heights are normally distributed with a known
mean $\mu = 68$ and some unknown variance $\sigma^2$.\\
\bigskip
\pause
Suppose before we see the data, we have a prior belief about the
distribution of $\sigma^2$.  Let our prior shape \red{$\alpha_0 = 5$} and
our prior scale \red{$\beta_0 = 20$.}
\medskip
\pause
\tiny
\begin{Schunk}
\begin{Sinput}
> alpha0 <- 5
> beta0 <- 20
\end{Sinput}
\end{Schunk}
\end{frame}

\begin{frame}[fragile]
Our posterior is a inverse gamma distribution with shape $\alpha_0 +
\frac{n}{2}$ and scale $\beta_0 + \frac{\sum_{i=1}^n (y_i-\mu)^2}{2}$
\medskip
\pause
\tiny
\begin{Schunk}
\begin{Sinput}
> alpha1 <- alpha0 + n/2
> beta1 <- beta0 + sum((heights - known.mean)^2)/2
> library(MCMCpack)
> posterior <- rinvgamma(10000, alpha1, beta1)
> post.mean <- mean(posterior)
> post.mean
\end{Sinput}
\begin{Soutput}
[1] 12.88139
\end{Soutput}
\begin{Sinput}
> post.var <- var(posterior)
> post.var
\end{Sinput}
\begin{Soutput}
[1] 3.136047
\end{Soutput}
\end{Schunk}
\normalsize
\pause
Hmm $\dots$ what if we increased our sample size?
\end{frame}

\begin{frame}[fragile]
\tiny
\begin{Schunk}
\begin{Sinput}
> n <- 1000
> heights <- rnorm(n, mean = known.mean, sd = sqrt(unknown.sigma.sq))
> alpha1 <- alpha0 + n/2
> beta1 <- beta0 + sum((heights - known.mean)^2)/2
> posterior <- rinvgamma(10000, alpha1, beta1)
> post.mean <- mean(posterior)
> post.mean
\end{Sinput}
\begin{Soutput}
[1] 15.92281
\end{Soutput}
\begin{Sinput}
> post.var <- var(posterior)
> post.var
\end{Sinput}
\begin{Soutput}
[1] 0.5058952
\end{Soutput}
\end{Schunk}
\end{frame}


\end{document}
