\documentclass[handout]{beamer}


\usetheme{default}
\usepackage{subfigure}
\usepackage{amsmath}
\usepackage{Sweave}
\usepackage{graphicx}
\usepackage{color}
\usepackage{amsfonts}
\usepackage{amssymb}
\usepackage{multicol}
\usepackage[all]{xy}
\usepackage{bm}


\author{Patrick Lam}
\title{MCMC Methods: Gibbs Sampling and the Metropolis-Hastings Algorithm}
\date{}
%\date{October 27, 2008}

\begin{document}

\newcommand{\red}{\textcolor{red}}
\newcommand{\blue}{\textcolor{blue}}
\newcommand{\purple}{\textcolor{purple}}

\frame{\titlepage}

\begin{frame}
\frametitle{Outline}
\tableofcontents
\end{frame}

\section{Introduction to Markov Chain Monte Carlo}

\begin{frame}
\frametitle{Outline}
\tableofcontents[currentsection]
\end{frame}

\begin{frame}
\frametitle{What is Markov Chain Monte Carlo (MCMC)?}
\pause
\textbf{Markov Chain}: a \textit{stochastic process} in which future states are
independent of past states given the present state\\
\bigskip
\pause
\textbf{Monte Carlo}: simulation\\
\bigskip
\pause
Up until now, we've done a lot of Monte Carlo simulation to find
integrals rather than doing it analytically, a process called \textit{Monte
Carlo Integration}.\\
\bigskip
\pause 
Basically a fancy way of saying we can take
quantities of interest of a distribution from simulated draws from the
distribution.
\end{frame}

\begin{frame}
\frametitle{Monte Carlo Integration}
\pause
Suppose we have a distribution $p(\theta)$ (perhaps a posterior) that
we want to take quantities of interest from.\\
\bigskip
\pause
To derive it analytically, we need to take integrals:
\begin{eqnarray*}
I = \int_{\Theta} g(\theta) p(\theta) d\theta
\end{eqnarray*}
where $g(\theta)$ is some function of $\theta$ ($g(\theta) =
\theta$ for the mean and $g(\theta) = (\theta - E(\theta))^2$ for the
variance).\\
\bigskip
\pause
We can approximate the integrals via Monte Carlo Integration by simulating
$M$ values from $p(\theta)$ and calculating
\begin{eqnarray*}
\hat{I}_M = \frac{1}{M} \sum_{i=1}^M g(\theta^{(i)})
\end{eqnarray*}
\end{frame}

\begin{frame}[fragile]
For example, we can compute the expected value of the Beta(3,3)
distribution analytically:
\pause
\begin{eqnarray*}
E(\theta) = \int_{\Theta} \theta p(\theta) d\theta =
\int_{\Theta} \theta \frac{\Gamma(6)}{\Gamma(3) \Gamma(3)} \theta^{2}
(1-\theta)^2 d\theta = \frac{1}{2}
\end{eqnarray*}
\pause
or via Monte Carlo methods:
\pause
\medskip
\tiny
\begin{Schunk}
\begin{Sinput}
> M <- 10000
> beta.sims <- rbeta(M, 3, 3)
> sum(beta.sims)/M
\end{Sinput}
\begin{Soutput}
[1] 0.5013
\end{Soutput}
\end{Schunk}
\pause
\normalsize
Our Monte Carlo approximation $\hat{I}_M$ is a simulation consistent
estimator of the true value $I$: \pause $\hat{I}_M
\rightarrow I$ as $M \rightarrow \infty$.\\
\bigskip
\pause
We know this to be true from the Strong Law of Large Numbers.
\end{frame}

\begin{frame}
\frametitle{Strong Law of Large Numbers (SLLN)}
\pause
Let $X_1, X_2, \dots$ be a sequence of \textbf{independent} and
identically distributed random variables, each having a finite mean
$\mu = E(X_i)$.\\
\bigskip
Then with probability 1,
\begin{eqnarray*}
\frac{X_1 + X_2 + \dots + X_M}{M} \rightarrow \mu \; \mathrm{as} \; M
\rightarrow \infty\\
\end{eqnarray*}
\pause
In our previous example, each simulation draw was \textbf{independent} and
distributed from the same Beta(3,3) distribution.\\
\bigskip
\pause
This also works with variances and other quantities of interest, since a
function of i.i.d.\ random variables are also i.i.d.\ random variables.\\
\bigskip
\pause
But what if we can't generate draws that are \textbf{independent}?

\end{frame}

\begin{frame}
Suppose we want to draw from our posterior distribution $p(\theta |
y)$, but we cannot sample independent draws from it.\\
\bigskip
\pause
For example, we often do not know the normalizing constant.\\
\bigskip
\pause
However, we may be able to sample draws from $p(\theta | y)$ that are
slightly dependent.\\
\bigskip
\pause
If we can sample slightly dependent draws using a \textbf{Markov
chain}, then we can still find quantities of interests from those draws.
\end{frame}

\begin{frame}
\frametitle{What is a Markov Chain?}
\pause
Definition: a \textit{stochastic process} in which future states are
independent of past states given the present state\\
\bigskip
\pause
Stochastic process: a \textit{consecutive} set of \textit{random} (not
deterministic) quantities defined on some known state space $\Theta$.
\pause
\begin{itemize}
\item think of $\Theta$ as our parameter space.
\pause
\item \textit{consecutive} implies a time component, indexed by $t$.
\end{itemize}
\pause
\bigskip
Consider a draw of $\bm{\theta}^{(t)}$ to be a state at iteration $t$.  The next
draw $\bm{\theta}^{(t+1)}$ is dependent only on the current draw
$\bm{\theta}^{(t)}$, and not on any past draws.\\
\pause 
\bigskip
This satisfies the \textbf{Markov property}:
\pause
\begin{eqnarray*}
p(\bm{\theta}^{(t+1)} | \bm{\theta}^{(1)}, \bm{\theta}^{(2)}, \dots,
\bm{\theta}^{(t)}) = p(\bm{\theta}^{(t+1)} | \bm{\theta}^{(t)})
\end{eqnarray*}
\end{frame}

\begin{frame}
So our Markov chain is a bunch of draws of $\bm{\theta}$ that are each
slightly dependent on the previous one.  \pause The chain wanders
around the parameter space, remembering only where it has been in the
last period.\\
\pause
\bigskip
What are the rules governing how the chain jumps from one state to
another at each period?\\
\pause
\bigskip
The jumping rules are governed by a \textbf{transition kernel}, which
is a mechanism that describes the probability of moving to some other
state based on the current state. 
\end{frame}

\begin{frame}
\frametitle{Transition Kernel}
\pause
For discrete state space ($k$ possible states): \pause a $k \times k$ matrix of transition probabilities.\\
\bigskip
\pause
Example: Suppose $k = 3$.  \pause The $3 \times 3$ transition matrix
$\bm{P}$ would be
\begin{table}[!htp]
\begin{center}
\begin{tabular}{|c|c|c|}
\hline
$p(\bm{\theta_A}^{(t+1)} | \bm{\theta_A}^{(t)} )$ & $p(\bm{\theta_B}^{(t+1)} |
\bm{\theta_A}^{(t)} )$ & $p(\bm{\theta_C}^{(t+1)} |
\bm{\theta_A}^{(t)} )$ \\
\hline
$p(\bm{\theta_A}^{(t+1)} | \bm{\theta_B}^{(t)} )$ & $p(\bm{\theta_B}^{(t+1)} |
\bm{\theta_B}^{(t)} )$ & $p(\bm{\theta_C}^{(t+1)} |
\bm{\theta_B}^{(t)} )$ \\
\hline
$p(\bm{\theta_A}^{(t+1)} | \bm{\theta_C}^{(t)} )$ & $p(\bm{\theta_B}^{(t+1)} |
\bm{\theta_C}^{(t)} )$ & $p(\bm{\theta_C}^{(t+1)} |
\bm{\theta_C}^{(t)} )$ \\
\hline
\end{tabular}
\end{center}
\end{table}
where the subscripts index the 3 possible values that $\bm{\theta}$
can take.  \\
\bigskip
\pause
The rows sum to one and define a conditional PMF, conditional on the
current state.  \pause The columns
are the marginal probabilities of being in a certain state in the next
period.\\
\pause
\bigskip
For continuous state space (infinite possible states), the transition
kernel is a bunch of conditional PDFs: $f(\bm{\theta_j}^{(t+1)} | \bm{\theta_i}^{(t)})$
\end{frame}

\begin{frame}
\frametitle{How Does a Markov Chain Work? (Discrete Example)}
\pause
\begin{enumerate}
\item Define a starting distribution $\prod^{(0)}$ (a $1 \times k$
vector of probabilities that sum to one).
\pause
\medskip
\item At iteration 1, our distribution $\prod^{(1)}$ (from which $\bm{\theta}^{(1)}$ is drawn) is
\begin{center}
\begin{tabular}{ccccc}
$\prod^{(1)}$ & = & $\prod^{(0)}$ & $\times$ & $\bm{P}$\\
$(1 \times k)$ & & $(1 \times k)$ & $\times$ & $(k \times k)$
\end{tabular}
\end{center}
\pause
\medskip
\item At iteration 2, our distribution $\prod^{(2)}$ (from which $\bm{\theta}^{(2)}$ is drawn) is
\begin{center}
\begin{tabular}{ccccc}
%$\prod^{(2)}$ & = & $\prod^{(0)}$ & $\times$ & $\bm{P}$ & $\times$ & $\bm{P}$\\
%$(1 \times k)$ & & $(1 \times k)$ & $\times$ & $(k \times k)$ &
%$\times$ & $(k \times k)$
$\prod^{(2)}$ & = & $\prod^{(1)}$ & $\times$ & $\bm{P}$\\
$(1 \times k)$ & & $(1 \times k)$ & $\times$ & $(k \times k)$
\end{tabular}
\end{center}
\medskip
\pause
\item At iteration $t$, our distribution $\prod^{(t)}$ (from which
$\bm{\theta}^{(t)}$ is drawn) is $\prod^{(t)}  = \prod^{(t-1)} \times
\; \bm{P} = \prod^{(0)}
\times \; \bm{P}^t$
\end{enumerate}
\end{frame}

\begin{frame}
\frametitle{Stationary (Limiting) Distribution}
\pause
Define a stationary distribution $\pi$ to be some distribution
$\prod$ such that $\pi = \pi \bm{P}$.  \\
\bigskip
\pause
For all the MCMC algorithms we use in Bayesian statistics, the Markov
chain will typically \textbf{converge} to $\pi$ regardless of our starting points.\\
\bigskip
\pause
So if we can devise a Markov chain whose stationary distribution $\pi$
is our desired posterior distribution $p(\bm{\theta} | y)$, then we
can run this chain to get draws that are approximately from
$p(\bm{\theta} | y)$ once the chain has converged.\\
\end{frame}

\begin{frame}
\frametitle{Burn-in}
\pause
Since convergence usually occurs regardless of our starting point, we
can usually pick any feasible (for example, picking starting draws
that are in the parameter space) starting point.\\
\bigskip
\pause 
However, the time it takes for the chain to converge varies depending
on the starting point.\\
\bigskip
\pause
As a matter of practice, most people throw out a certain number of the
first draws, known as the \textbf{burn-in}.  This is to make our draws
closer to the stationary distribution and less dependent on the
starting point.\\
\bigskip
\pause
However, it is unclear how much we should \textbf{burn-in} since our
draws are all slightly dependent and we don't know exactly when
convergence occurs.  
\end{frame}

\begin{frame}
\frametitle{Monte Carlo Integration on the Markov Chain}
\pause
Once we have a Markov chain that has converged to the stationary
distribution, then the draws in our chain appear to be like draws from
$p(\bm{\theta} | y)$, \pause so it seems like we should be able to use Monte
Carlo Integration methods to find quantities of interest.\\
\bigskip
\pause
One problem: our draws are not independent, which we required for
Monte Carlo Integration to work (remember SLLN).\\
\bigskip
\pause
Luckily, we have the \textbf{Ergodic Theorem}.
\end{frame}

\begin{frame}
\frametitle{Ergodic Theorem}
\pause
Let $\bm{\theta}^{(1)},\bm{\theta}^{(2)}, \dots, \bm{\theta}^{(M)}$ be
$M$ values from a Markov chain that is \textit{aperiodic},
\textit{irreducible}, and \textit{positive recurrent} (then the chain
is ergodic), and $E[g(\bm{\theta})] < \infty$.\\
\bigskip
\pause
Then with probability 1,
\begin{eqnarray*}
\frac{1}{M} \sum^{M}_{i=1} g(\bm{\theta}_i) \rightarrow \int_{\Theta}
g(\bm{\theta}) \pi(\bm{\theta}) d\bm{\theta}
\end{eqnarray*}
as $M \rightarrow \infty$, where $\pi$ is the stationary distribution.\\
\bigskip
\pause
This is the Markov chain analog to the SLLN, \pause and it allows us to
ignore the dependence between draws of the Markov chain when we
calculate quantities of interest from the draws.\\
\bigskip
\pause
But what does it mean for a chain to be \textit{aperiodic},
\textit{irreducible}, and \textit{positive recurrent}, and therefore ergodic? 
\end{frame}

\begin{frame}
\frametitle{Aperiodicity}
\pause
A Markov chain is \textbf{aperiodic} if the only length of time for
which the chain repeats some cycle of values is the trivial case with
cycle length equal to one.\\
\pause
\bigskip
Let A, B, and C denote the states (analogous to the possible values of
$\bm{\theta}$) in a 3-state Markov chain.  \pause The following chain
is \textit{periodic} with period 3, where the period is the number of
steps that it takes to return to a certain state.
\pause
$$
\xymatrix{
*++[o][F-]{A} \ar@(ul,ur)^0 \ar[rr]^1 && *++[o][F-]{B}
\ar@(ul,ur)^0 \ar[rr]^1 && *++[o][F-]{C} \ar@(ul,ur)^0 \ar@/^2pc/[llll]^1
}
$$
\pause
As long as the chain is not repeating an identical cycle, then the
chain is \textbf{aperiodic}.
\end{frame}

\begin{frame}
\frametitle{Irreducibility}
\pause
A Markov chain is \textbf{irreducible} if it is possible go from any state to
any other state (not necessarily in one step).\\
\bigskip
\pause
The following chain is \textit{reducible}, or not irreducible.
\pause
$$
\xymatrix{
*++[o][F-]{A} \ar@(ul,ur)^{0.5} \ar[rr]^{0.5} && *++[o][F-]{B}
\ar@(ul,ur)^{0.7} \ar@/^1pc/[rr]^{0.3} && *++[o][F-]{C} \ar@(ul,ur)^{0.4} \ar@/^1pc/[ll]^{0.6}
}
$$
\pause
The chain is not irreducible because we cannot get to A from B or C
regardless of the number of steps we take.
\end{frame}

\begin{frame}
\frametitle{Positive Recurrence}
\pause
A Markov chain is \textit{recurrent} if for any given state $i$, if
the chain starts at $i$, it will eventually return to $i$ with
probability 1. \\
\bigskip
\pause
A Markov chain is \textbf{positive recurrent} if the expected return
time to state $i$ is finite; \pause otherwise it is \textit{null recurrent}.\\
\pause
\bigskip
\bigskip
So if our Markov chain is \textbf{aperiodic}, \textbf{irreducible},
and \textbf{positive recurrent} (all the ones we use in
Bayesian statistics usually are), then it is ergodic and the ergodic
theorem allows us to do Monte Carlo Integration by calculating quantities of interest from our draws,
ignoring the dependence between draws.
\end{frame}

\begin{frame}
\frametitle{Thinning the Chain}
\pause
In order to break the dependence between draws in the Markov chain,
some have suggested only keeping every $d$th draw of the chain. \\
\bigskip
\pause
This is known as \textbf{thinning}.\\
\bigskip
\pause
Pros: 
\begin{itemize}
\pause
\item Perhaps gets you a little closer to i.i.d.\ draws.
\pause
\item Saves memory since you only store a fraction of the draws.
\end{itemize}
\bigskip
\pause
Cons:
\begin{itemize}
\pause
\item Unnecessary with ergodic theorem.
\pause
\item Shown to increase the variance of your Monte Carlo estimates.
\end{itemize}
\end{frame}

\begin{frame}
\frametitle{So Really, What is MCMC?}
\pause
\blue{MC}\red{MC} is a class of methods in which we can \red{simulate
draws} that are \blue{slightly dependent} and are
approximately from a (posterior) distribution.  \\
\bigskip
\pause
We then take those draws and calculate quantities of interest for the
(posterior) distribution.\\
\bigskip
\pause
In Bayesian statistics, there are generally two MCMC algorithms that
we use: the Gibbs Sampler and the Metropolis-Hastings algorithm.
\end{frame}

\section{Gibbs Sampling}

\begin{frame}
\frametitle{Outline}
\tableofcontents[currentsection]
\end{frame}

\begin{frame}
\frametitle{Gibbs Sampling}
\pause
Suppose we have a joint distribution $p(\theta_1, \dots,\theta_k)$
that we want to sample from (for example, a posterior distribution). \\  
\bigskip
\pause
We can use the Gibbs sampler to sample from the joint distribution if
we knew the \textbf{full conditional} distributions for each parameter.\\
\bigskip
\pause
For each parameter, the \textbf{full conditional} distribution is the
distribution of the parameter conditional on the known information and
all the other parameters: \pause $p(\theta_j | \theta_{-j}, y)$\\
\bigskip
\pause
How can we know the joint distribution simply by knowing the full
conditional distributions?
\end{frame}

\begin{frame}
\frametitle{The Hammersley-Clifford Theorem (for two blocks)}
\pause
Suppose we have a joint density $f(x,y)$.  \pause The theorem proves
that we can write out the joint density in terms of the conditional
densities $f(x|y)$ and $f(y|x)$:
\pause
\begin{eqnarray*}
f(x,y) = \frac{\red{f(y|x)}}{\blue{\int \frac{f(y|x)}{f(x|y)} dy}}
\end{eqnarray*}
\pause
We can write the denominator as
\color{blue}
\begin{eqnarray*}
\int \frac{f(y|x)}{f(x|y)} dy &=& \int
\frac{\frac{f(x,y)}{f(x)}}{\frac{f(x,y)}{f(y)}} dy \\
\pause
&=& \int \frac{f(y)}{f(x)} dy\\
\pause
&=& \frac{1}{f(x)}
\end{eqnarray*}
\color{black}
\end{frame}

\begin{frame}
Thus, our right-hand side is
\begin{eqnarray*}
\frac{\red{f(y|x)}}{\blue{\frac{1}{f(x)}}} &=& \red{f(y|x)} \blue{f(x)}\\
\pause
&=& f(x,y)
\end{eqnarray*}
\pause
The theorem shows that knowledge of the conditional densities allows
us to get the joint density.  \\
\pause
\bigskip
This works for more than two blocks of parameters. \\
\pause
\bigskip
But how do we figure out the full conditionals?
\end{frame}

\begin{frame}
\frametitle{Steps to Calculating Full Conditional Distributions}
\pause
Suppose we have a posterior $p(\bm{\theta} | \bm{y})$.  \pause
To calculate the full conditionals for each $\theta$, do the following:
\pause
\bigskip
\begin{enumerate}
\item Write out the full posterior ignoring constants of proportionality.
\pause
\item Pick a block of parameters (for example, $\theta_1$) and drop
everything that doesn't depend on $\theta_1$.
\pause
\item Use your knowledge of distributions to figure out what the
normalizing constant is (and thus what the full
conditional distribution $p(\theta_1 | \theta_{-1}, \bm{y})$ is).
\pause
\item Repeat steps 2 and 3 for all parameter blocks.
\end{enumerate}
\end{frame}

\begin{frame}
\frametitle{Gibbs Sampler Steps}
\pause
Let's suppose that we are interested in sampling from the posterior
$p(\bm{\theta} | \bm{y})$, where $\bm{\theta}$ is a vector of three
parameters, $\theta_1, \theta_2, \theta_3$.\\
\bigskip
\pause
The steps to a Gibbs Sampler \tiny{(and the analogous steps in
the MCMC process)} \normalsize are
\bigskip
\pause
\begin{enumerate}
\item Pick a vector of starting values $\bm{\theta}^{(0)}$.
\tiny{(Defining a starting distribution
$\prod^{(0)}$ and drawing $\bm{\theta}^{(0)}$ from it.)} \normalsize
\pause
\medskip
\item Start with any $\theta$ (order does not matter, but I'll start
with $\theta_1$ for convenience). \pause Draw a value $\theta_1^{(1)}$
from the full conditional $p(\theta_1 | \theta_2^{(0)}, \theta_3^{(0)}, \bm{y})$.
\end{enumerate}
\end{frame}

\begin{frame}
\begin{enumerate}
\item [3.] Draw a value $\theta_2^{(1)}$ (again order does not matter) from
the full conditional  $p(\theta_2 | \theta_1^{(1)}, \theta_3^{(0)},
\bm{y})$. \pause Note that we must use the updated value of $\theta_1^{(1)}$.
\pause
\medskip
\item [4.] Draw a value $\theta_3^{(1)}$ from the full conditional
$p(\theta_3 | \theta_1^{(1)}, \theta_2^{(1)}, \bm{y})$ using both
updated values. \pause \tiny{(Steps 2-4 are analogous to multiplying
$\prod^{(0)}$ and $\bm{P}$ to get $\prod^{(1)}$ and then drawing
$\bm{\theta}^{(1)}$ from $\prod^{(1)}$.)} \normalsize
\pause
\medskip
\item [5.] Draw $\bm{\theta}^{(2)}$ using $\bm{\theta}^{(1)}$ and
continually using the most updated values.
\pause
\medskip
\item [6.] Repeat until we get $M$ draws, with each draw being a vector $\bm{\theta}^{(t)}$.
\pause
\medskip
\item [7.] Optional burn-in and/or thinning.
\end{enumerate}
\pause
\bigskip
Our result is a Markov chain with a bunch of draws of $\bm{\theta}$
that are approximately from our posterior.  \pause We can do Monte
Carlo Integration on
those draws to get quantities of interest.
\end{frame}

\begin{frame}[fragile]
\frametitle{An Example (Robert and Casella, 10.17)\footnote{\tiny
{Robert, Christian P. and George Casella.  2004.  \textit{Monte Carlo
Statistical Methods, 2nd edition.}  Springer.}}}
\pause
\normalsize
Suppose we have data of the number of failures ($y_i$) for each of 10 pumps in
a nuclear plant. \\
\pause
\bigskip 
We also have the times ($t_i$) at which each pump was observed.
\medskip
\pause
\tiny
\begin{Schunk}
\begin{Sinput}
> y <- c(5, 1, 5, 14, 3, 19, 1, 1, 4, 22)
> t <- c(94, 16, 63, 126, 5, 31, 1, 1, 2, 10)
> rbind(y, t)
\end{Sinput}
\begin{Soutput}
  [,1] [,2] [,3] [,4] [,5] [,6] [,7] [,8] [,9] [,10]
y    5    1    5   14    3   19    1    1    4    22
t   94   16   63  126    5   31    1    1    2    10
\end{Soutput}
\end{Schunk}
\normalsize
\medskip
\pause
We want to model the number of failures with a Poisson likelihood,
where the expected number of failure $\lambda_i$ differs for each
pump.  \pause  Since the time which we observed each pump is
different, we need to scale each $\lambda_i$ by its observed time $t_i$.\\
\pause
\bigskip
Our likelihood is $\prod_{i=1}^{10} \mathrm{Poisson}(\lambda_i t_i)$.
\end{frame}

\begin{frame}
Let's put a \red{Gamma($\alpha, \beta$)} prior on $\lambda_i$ with
$\alpha = 1.8$, so the $\lambda_i$s are drawn from the same distribution.  \\
\pause
\bigskip
Also, let's put a \red{Gamma($\gamma, \delta$)} prior on $\beta$, with
$\gamma = 0.01$ and $\delta = 1$.\\
\pause
\bigskip 
So our model has 11 parameters that are unknown (10 $\lambda_i$s and $\beta$).\\
\pause
\bigskip
Our posterior is 
\begin{eqnarray*}
\blue{p(\bm{\lambda}, \beta | \bm{y}, \bm{t})} &\propto& \left( \prod_{i=1}^{10} \mathrm{Poisson}(\lambda_i
t_i) \times  \red{\mathrm{Gamma}(\alpha, \beta)} \right) \red{\times
\mathrm{Gamma}(\gamma, \delta)} \\
\pause
&=& \left( \prod_{i=1}^{10} \frac{e^{-\lambda_i t_i} (\lambda_i
t_i)^{y_i}}{y_i !} \times \red{\frac{\beta^{\alpha}}{\Gamma (\alpha)}
\lambda_i^{\alpha-1} e^{-\beta \lambda_i}} \right) \\
&\phantom{=}& \red{\times \frac{\delta^{\gamma}}{\Gamma (\gamma)}
\beta^{\gamma-1} e^{-\delta \beta} }
\end{eqnarray*}
\end{frame}

\begin{frame}
\begin{eqnarray*}
\blue{p(\bm{\lambda}, \beta | \bm{y}, \bm{t})} &\propto& \left(
\prod_{i=1}^{10} e^{-\lambda_i t_i} (\lambda_i t_i)^{y_i} \times
\red{\beta^{\alpha} \lambda_i^{\alpha-1} e^{-\beta \lambda_i}} \right) \red{\times 
\beta^{\gamma-1} e^{-\delta \beta} }\\
\pause
&=& \left( \prod_{i=1}^{10} \lambda_i^{y_i + \alpha - 1} e^{-(t_i +
\beta)\lambda_i} \right) \beta^{10\alpha + \gamma - 1} e^{-\delta \beta}\\
\end{eqnarray*}
\pause
\medskip
Finding the full conditionals:
\pause
\begin{eqnarray*}
p(\lambda_i | \lambda_{-i}, \beta, \bm{y}, \bm{t}) &\propto& \lambda_i^{y_i + \alpha - 1} e^{-(t_i +
\beta)\lambda_i} \\
\pause
p(\beta | \bm{\lambda}, \bm{y}, \bm{t}) &\propto&  e^{-\beta
(\delta + \sum_{i=1}^{10} \lambda_i)} \beta^{10\alpha + \gamma - 1}\\
\end{eqnarray*}
\pause
$p(\lambda_i | \lambda_{-i}, \beta, \bm{y}, \bm{t})$ is a Gamma($y_i + \alpha, t_i + \beta$) distribution.\\
\pause
\bigskip
$p(\beta | \bm{\lambda}, \bm{y}, \bm{t})$ is a Gamma($10\alpha + \gamma, \delta + \sum_{i=1}^{10}
\lambda_i$) distribution.
\end{frame}

\begin{frame}[fragile]
\frametitle{Coding the Gibbs Sampler}
\pause
\begin{enumerate}
\item Define starting values for $\beta$ \pause (we only need to define
$\beta$ here because we will draw $\bm{\lambda}$ first and it only
depends on $\beta$ and other given values).
\pause
\tiny
\medskip
\begin{Schunk}
\begin{Sinput}
> beta.cur <- 1
\end{Sinput}
\end{Schunk}
\bigskip
\normalsize
\pause
\item Draw $\bm{\lambda}^{(1)}$ from its full conditional \pause (we're
drawing all the $\lambda_i$s as a block because they all only depend
on $\beta$ and not each other). 
\tiny
\medskip
\pause
\begin{Schunk}
\begin{Sinput}
> lambda.update <- function(alpha, beta, y, t) {
+     rgamma(length(y), y + alpha, t + beta)
+ }
\end{Sinput}
\end{Schunk}
\normalsize
\bigskip
\pause
\item Draw $\beta^{(1)}$ from its full conditional, using $\bm{\lambda}^{(1)}$.
\medskip
\tiny
\pause
\begin{Schunk}
\begin{Sinput}
> beta.update <- function(alpha, gamma, delta, lambda, y) {
+     rgamma(1, length(y) * alpha + gamma, delta + sum(lambda))
+ }
\end{Sinput}
\end{Schunk}
\normalsize
\end{enumerate}

\end{frame}

\begin{frame}[fragile]
\begin{enumerate}
\item [4.] Repeat using most updated values until we get $M$ draws.
\pause
\bigskip
\item [5.] Optional burn-in and thinning.
\bigskip
\pause
\item [6.] Make it into a function.
\end{enumerate}
\end{frame}

\begin{frame}[fragile]
\tiny
\begin{Schunk}
\begin{Sinput}
> gibbs <- function(n.sims, beta.start, alpha, gamma, delta, 
+     y, t, burnin = 0, thin = 1) {
+     beta.draws <- c()
+     lambda.draws <- matrix(NA, nrow = n.sims, ncol = length(y))
+     beta.cur <- beta.start
+     lambda.update <- function(alpha, beta, y, t) {
+         rgamma(length(y), y + alpha, t + beta)
+     }
+     beta.update <- function(alpha, gamma, delta, lambda, 
+         y) {
+         rgamma(1, length(y) * alpha + gamma, delta + sum(lambda))
+     }
+     for (i in 1:n.sims) {
+         lambda.cur <- lambda.update(alpha = alpha, beta = beta.cur, 
+             y = y, t = t)
+         beta.cur <- beta.update(alpha = alpha, gamma = gamma, 
+             delta = delta, lambda = lambda.cur, y = y)
+         if (i > burnin & (i - burnin)%%thin == 0) {
+             lambda.draws[(i - burnin)/thin, ] <- lambda.cur
+             beta.draws[(i - burnin)/thin] <- beta.cur
+         }
+     }
+     return(list(lambda.draws = lambda.draws, beta.draws = beta.draws))
+ }
\end{Sinput}
\end{Schunk}

\end{frame}

\begin{frame}[fragile]
\begin{enumerate}
\item [7.] Do Monte Carlo Integration on the resulting Markov chain, which are
samples approximately from the posterior.
\medskip
\pause
\tiny
\begin{Schunk}
\begin{Sinput}
> posterior <- gibbs(n.sims = 10000, beta.start = 1, alpha = 1.8, 
+     gamma = 0.01, delta = 1, y = y, t = t)
> colMeans(posterior$lambda.draws)
\end{Sinput}
\begin{Soutput}
 [1] 0.07113 0.15098 0.10447 0.12321 0.65680 0.62212 0.86522 0.85465
 [9] 1.35524 1.92694
\end{Soutput}
\begin{Sinput}
> mean(posterior$beta.draws)
\end{Sinput}
\begin{Soutput}
[1] 2.389
\end{Soutput}
\begin{Sinput}
> apply(posterior$lambda.draws, 2, sd)
\end{Sinput}
\begin{Soutput}
 [1] 0.02759 0.08974 0.04012 0.03071 0.30899 0.13676 0.55689 0.54814
 [9] 0.60854 0.40812
\end{Soutput}
\begin{Sinput}
> sd(posterior$beta.draws)
\end{Sinput}
\begin{Soutput}
[1] 0.6986
\end{Soutput}
\end{Schunk}
\end{enumerate}
\end{frame}

\section{The Metropolis-Hastings Algorithm}

\begin{frame}
\frametitle{Outline}
\tableofcontents[currentsection]
\end{frame}

\begin{frame}
Suppose we have a posterior $\blue{p(\bm{\theta} | \bm{y})}$ that we want to
sample from, but
\pause
\begin{itemize}
\item the posterior doesn't look like any distribution we know (no conjugacy)
\pause
\item the posterior consists of more than 2 parameters (grid
approximations intractable)
\pause
\item some (or all) of the full conditionals do not look like any
distributions we know (no Gibbs sampling for those whose full
conditionals we don't know)
\end{itemize}
\pause
\bigskip
If all else fails, we can use the \textbf{Metropolis-Hastings}
algorithm, which will always work.
\end{frame}

\begin{frame}
\frametitle{Metropolis-Hastings Algorithm}
\pause
The Metropolis-Hastings Algorithm follows the following steps:
\pause
\bigskip
\begin{enumerate}
\item Choose a starting value $\bm{\theta}^{(0)}$.
\pause
\item At iteration $t$, draw a candidate $\bm{\theta}^*$ from a jumping
distribution $J_t(\bm{\theta}^* | \bm{\theta}^{(t-1)})$.
\pause
\item Compute an acceptance ratio (probability):
\begin{eqnarray*}
r = \frac{p(\bm{\theta}^* | \bm{y})/J_t(\bm{\theta}^* | \bm{\theta}^{(t-1)})}{p(\bm{\theta}^{(t-1)} | \bm{y})/J_t(\bm{\theta}^{(t-1)} | \bm{\theta}^*)}
\end{eqnarray*}
\pause
\item Accept $\bm{\theta}^*$ as $\bm{\theta}^{(t)}$ with probability
min($r,1$).  If $\bm{\theta}^*$ is not accepted, then $\bm{\theta}^{(t)} = \bm{\theta}^{(t-1)}$.
\pause
\item Repeat steps 2-4 $M$ times to get $M$ draws from $\blue{p(\bm{\theta}
| \bm{y})}$, with optional burn-in and/or thinning.
\end{enumerate}
\end{frame}

\begin{frame}
\frametitle{Step 1: Choose a starting value $\bm{\theta}^{(0)}$.}
\pause
This is equivalent to drawing from our initial stationary distribution.\\
\bigskip
\pause
The important thing to remember is that $\bm{\theta}^{(0)}$ must have
positive probability.
\begin{eqnarray*}
p(\bm{\theta}^{(0)} | \bm{y}) > 0
\end{eqnarray*}
\bigskip
\pause
Otherwise, we are starting with a value that cannot be drawn.
\end{frame}

\begin{frame}
\frametitle{Step 2: Draw $\bm{\theta}^*$ from $J_t(\bm{\theta}^* | \bm{\theta}^{(t-1)})$.}
\pause
The jumping distribution  $J_t(\bm{\theta}^* | \bm{\theta}^{(t-1)})$
determines where we move to in the next iteration of the Markov chain
(analogous to the transition kernel).
\pause
The support of the jumping distribution must contain the support of
the posterior. \\
\pause
\bigskip
The original \textbf{Metropolis algorithm} required that
$J_t(\bm{\theta}^* | \bm{\theta}^{(t-1)})$ be a symmetric
distribution (such as the normal distribution), \pause that is
\begin{eqnarray*}
 J_t(\bm{\theta}^* | \bm{\theta}^{(t-1)}) = J_t(\bm{\theta}^{(t-1)} | \bm{\theta}^*)
\end{eqnarray*}
\pause
We now know with the Metropolis-Hastings algorithm that symmetry is
unnecessary.  \\
\bigskip
\pause
If we have a symmetric jumping distribution that is dependent on
$\bm{\theta}^{(t-1)}$, then we have what is known as \textbf{random walk Metropolis sampling}.
\end{frame}

\begin{frame}
If our jumping distribution does not depend on $\bm{\theta}^{(t-1)}$, 
\pause
\begin{eqnarray*}
J_t(\bm{\theta}^* | \bm{\theta}^{(t-1)}) = J_t(\bm{\theta}^*)
\end{eqnarray*}
\pause
then we have what is known as \textbf{independent Metropolis-Hastings sampling}.\\
\bigskip
\pause
Basically all our candidate draws $\bm{\theta}^*$ are drawn from the
same distribution, regardless of where the previous draw was.\\
\bigskip
\pause
This can be extremely efficient or extremely inefficient, depending on
how close the jumping distribution is to the posterior.\\
\bigskip
\pause
Generally speaking, chain will behave well only if the jumping
distribution has heavier tails than the posterior.
\end{frame}

\begin{frame}
\frametitle{Step 3: Compute acceptance ratio $r$.}
\pause
\begin{eqnarray*}
r = \frac{p(\bm{\theta}^* | \bm{y})/J_t(\bm{\theta}^* | \bm{\theta}^{(t-1)})}{p(\bm{\theta}^{(t-1)} | \bm{y})/J_t(\bm{\theta}^{(t-1)} | \bm{\theta}^*)}
\end{eqnarray*}
\pause
In the case where our jumping distribution is symmetric,
\begin{eqnarray*}
r = \frac{p(\bm{\theta}^* | \bm{y})}{p(\bm{\theta}^{(t-1)} | \bm{y})}
\end{eqnarray*}
\pause
If our candidate draw has higher probability than our current draw,
then our candidate is better so we definitely accept it.  \pause
Otherwise, our candidate is accepted according to the ratio of the
probabilities of the candidate and current draws. \\
\bigskip
\pause
Note that since $r$ is a ratio, we only need $\blue{p(\bm{\theta} |
\bm{y})}$ \textit{up to a constant of proportionality} since
$p(\bm{y})$ cancels out in both the numerator and denominator.
\end{frame}

\begin{frame}
In the case where our jumping distribution is not symmetric,
\begin{eqnarray*}
r = \frac{p(\bm{\theta}^* | \bm{y})/J_t(\bm{\theta}^* | \bm{\theta}^{(t-1)})}{p(\bm{\theta}^{(t-1)} | \bm{y})/J_t(\bm{\theta}^{(t-1)} | \bm{\theta}^*)}
\end{eqnarray*}
\pause
We need to weight our evaluations of the draws at the posterior
densities by how likely we are to draw each draw.  \\
\bigskip
\pause
For example, if we are very likely to jump to some $\bm{\theta}^*$,
then $J_t(\bm{\theta}^* | \bm{\theta}^{(t-1)})$ is likely to be high,
so we should accept less of them than some other $\bm{\theta}^*$ that
we are less likely to jump to.\\
\pause
\bigskip
In the case of independent Metropolis-Hastings sampling,
\begin{eqnarray*}
r = \frac{p(\bm{\theta}^* | \bm{y})/J_t(\bm{\theta}^*)}{p(\bm{\theta}^{(t-1)} | \bm{y})/J_t(\bm{\theta}^{(t-1)})}
\end{eqnarray*}

\end{frame}

\begin{frame}
\frametitle{Step 4: Decide whether to accept $\bm{\theta}^*$.}
\pause
Accept $\bm{\theta}^*$ as $\bm{\theta}^{(t)}$ with probability
min($r,1$).  If $\bm{\theta}^*$ is not accepted, then
$\bm{\theta}^{(t)} = \bm{\theta}^{(t-1)}$. \\
\bigskip
\pause
\begin{enumerate}
\item For each $\bm{\theta}^*$, draw a value $u$ from the Uniform(0,1)
distribution.
\pause
\item If $u \le r$, accept $\bm{\theta}^*$ as $\bm{\theta}^{(t)}$.
Otherwise, use $\bm{\theta}^{(t-1)}$ as $\bm{\theta}^{(t)}$
\end{enumerate}
\pause
\bigskip
Candidate draws with higher density than the current draw are always accepted.\\
\bigskip
\pause
Unlike in rejection sampling, each iteration always produces a draw,
either $\bm{\theta}^*$ or $\bm{\theta}^{(t-1)}$.
\end{frame}

\begin{frame}
\frametitle{Acceptance Rates}
\pause
It is important to monitor the \textit{acceptance rate} (the fraction of
candidate draws that are accepted) of your Metropolis-Hastings algorithm.\\
\pause
\bigskip
If your acceptance rate is too high, the chain is probably not mixing
well (not moving around the parameter space quickly enough).\\
\pause
\bigskip
If your acceptance rate is too low, your algorithm is too inefficient
(rejecting too many candidate draws).\\
\pause
\bigskip
What is too high and too low depends on your specific algorithm, but generally
\pause
\begin{itemize}
\item random walk: somewhere between 0.25 and 0.50 is recommended
\pause
\item independent: something close to 1 is preferred
\end{itemize}
\end{frame}

\begin{frame}[fragile]
\frametitle{A Simple Example}
\pause
Using a random walk Metropolis algorithm to sample from a Gamma(1.7,
4.4) distribution with a Normal jumping distribution with standard
deviation of 2.
\medskip
\pause
\tiny
\begin{Schunk}
\begin{Sinput}
> mh.gamma <- function(n.sims, start, burnin, cand.sd, shape, rate) {
+     theta.cur <- start
+     draws <- c()
+     theta.update <- function(theta.cur, shape, rate) {
+         theta.can <- rnorm(1, mean = theta.cur, sd = cand.sd)
+         accept.prob <- dgamma(theta.can, shape = shape, rate = rate)/dgamma(theta.cur, 
+             shape = shape, rate = rate)
+         if (runif(1) <= accept.prob) 
+             theta.can
+         else theta.cur
+     }
+     for (i in 1:n.sims) {
+         draws[i] <- theta.cur <- theta.update(theta.cur, shape = shape, 
+             rate = rate)
+     }
+     return(draws[(burnin + 1):n.sims])
+ }
> mh.draws <- mh.gamma(10000, start = 1, burnin = 1000, cand.sd = 2, 
+     shape = 1.7, rate = 4.4)
\end{Sinput}
\end{Schunk}
\end{frame}


\end{document}
