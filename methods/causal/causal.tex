\documentclass{beamer}


\usetheme{default}
\usepackage{subfigure}
\usepackage{amsmath}
\usepackage{Sweave}
\usepackage{graphicx}
%\usepackage{color}
\usepackage{multicol}
\usepackage{bm}
\usepackage[all]{xy}


\author{Patrick Lam}
\title{Causal Inference}
\date{}
%\date{April 2, 2009}

\begin{document}

\newcommand{\red}{\textcolor{red}}
\newcommand{\blue}{\textcolor{blue}}
\newcommand{\purple}{\textcolor{purple}}

\frame{\titlepage}

\begin{frame}
\frametitle{Reference on History of Causal Inference}
Holland, Paul W. 1986.  ``Statistics and Causal Inference.''  {\it
Journal of the American Statistical Association} 81(396): 945-960. 
\end{frame}

\begin{frame}
\frametitle{What is a Causal Effect?}
\pause
\begin{eqnarray*}
Y_i(1) - Y_i(0)
\end{eqnarray*}
\pause
where $Y_i(1) = Y_i(T_i = 1)$ for some treatment variable $T$. \\
\bigskip
\bigskip
\pause
$Y_i(1)$ and $Y_i(0)$ are {\it potential outcomes} \pause in that they
represent the outcomes for individual $i$ had they received the
treatment or control respectively. \\
\bigskip
\pause
The {\bf fundamental problem of causal inference} is that only one of
$Y_i(1)$ and $Y_i(0)$ is observed, so we can never find the true
causal effect.\\
\pause
\bigskip
The approach we will discuss is known as the {\it Rubin Causal Model}.
\end{frame}

\begin{frame}
\frametitle{}
Instead of the individual treatment effect, we might be interested in
the {\bf average treatment effect (ATE)}: 
\pause
\begin{eqnarray*}
E[Y(1) - Y(0)] = E[Y(1)] - E[Y(0)]
\end{eqnarray*}
\pause
We cannot find the ATE because of the unobserved potential outcomes. \\
\pause
\bigskip
We might also be interested in the {\bf average treatment effect on
the treated (ATT)}:
\pause
\begin{eqnarray*}
E[Y(1|T=1) - \red{Y(0 | T=1)}] = E[Y_t(1) - \red{Y_t(0)}]
\end{eqnarray*}
\pause
We cannot find the ATT because of \red{unobserved} potential outcomes.
\end{frame}

\begin{frame}
\frametitle{Causal Inference as Missing Data Problem}
\pause
\begin{table}[!htp]
\begin{center}
\begin{tabular}{c|ccc|c}
\hline
$i$ & $T_i$ & $Y_i(0)$ & $Y_i(1)$ & $Y_i(1) - Y_i(0)$ \\
\hline
1 & 0 & 3 & \red{5} & \red{2}\\
2 & 1 & \red{2} & 5 & \red{3} \\
3 & 1 & \red{5} & 4 & \red{-1}\\
4 & 0 & 2 & \red{7} & \red{5}\\
5 & 1 & \red{1} & 2 & \red{1}\\
\hline
\end{tabular}
\end{center}
\end{table}
\pause
\bigskip
Some potential outcomes are \red{unobserved}, as are the ATE and ATT. \\
\pause
\bigskip
\red{ATE $=$ \pause 2} \\
\pause
\bigskip
\red{ATT $=$ \pause 1}
\end{frame}

\begin{frame}
\frametitle{Estimating ATE}
\pause
We can estimate the ATE in the following way:
\pause
\begin{eqnarray*}
\hat{\mathrm{ATE}} &=& E[Y_t(1) - Y_c(0)] \\
\pause
&=& E[Y_t(1)] - E[Y_c(0)] \\ 
\end{eqnarray*}
\pause
Both quantities are observed. \\
\pause
\bigskip
We basically find the average $Y$ for observations that received
treatment and average $Y$ for observations that received control. \\
\pause
\bigskip
What assumptions do we need for this estimate to be unbiased?
\pause
\begin{itemize}
\item SUTVA
\pause
\item unconfoundedness/ignorability
\end{itemize}
\end{frame}

\begin{frame}
\frametitle{Stable Unit Treatment Value Assumption}
\pause
The {\bf stable unit treatment value assumption (SUTVA)} assumes that
\pause
\begin{itemize}
\item the treatment status of any unit does not affect the potential
outcomes of the other units (non-interference)
\pause
\item the treatments for all units are comparable (no variation in treatment) 
\end{itemize}
\bigskip
\pause
Violations: 
\pause
\begin{itemize}
\item Job training for too many people may flood the market with
qualified job applicants (interference) 
\pause
\item Some patients get extra-strength aspirin (variation in treatment)
\end{itemize}
\end{frame}

\begin{frame}
\frametitle{Ignorability/Unconfoundedness}
Unconfoundedness (strong ignorability): 
\begin{eqnarray*}
(Y(1), Y(0)) \bot T
\end{eqnarray*}
\pause
\bigskip
Treatment assignment is independent of the outcomes ($Y$). \\
\pause
\bigskip
Ignorability and Unconfoundedness are often used interchangeably. 
Technically, unconfoundedness is a stronger assumption.  Most people
just say ignorability.\\
\bigskip
\pause
Violations:
\pause
\begin{itemize}
\item Omitted Variable Bias
\end{itemize}
\end{frame}

\begin{frame}
\frametitle{Classical Randomized Experiment}
\pause
The gold standard of scientific research.\\
\pause
\bigskip
\begin{enumerate}
\item Randomly sample units from population.\\
\pause
\item Randomly assign treatment and control to the units.\\
\pause
\item Estimate ATE.
\end{enumerate}
\pause
\bigskip
SUTVA: can theoretically control for treatment variation and
non-interference \\
\pause
\bigskip
Ignorability: controlled for by random treatment assignment
\end{frame}

\begin{frame}
Possible problems:
\pause
\bigskip
\begin{itemize}
\item Compliance to treatment assignment:
\pause
\begin{itemize}
\item Never-taker: Unit never takes treatment
\pause
\item Always-taker: Unit always takes treatment
\pause
\item Complier: Unit takes treatment when assigned and control when
not assigned
\pause
\item Defier: Unit takes treatment when not assigned and control when assigned
\end{itemize}
\pause
\bigskip
\item Unlucky random treatment assignment violates ignorability
\pause
\begin{itemize}
\item Problem goes away as sample size gets large.
\end{itemize}
\end{itemize}
\end{frame}

\begin{frame}
\frametitle{Observational Data}
\pause
We have a dataset where we only observe after the experiment occurred
and we have no control over treatment assignment. \\
\pause
\bigskip
This is the case with most of the sciences.
\pause
\bigskip
\begin{enumerate}
\item Gather dataset.
\pause
\item Estimate ATE or ATT with a model.
\end{enumerate}
\pause
\bigskip
SUTVA: assumed (a problematic assumption most of the time) \\
\pause
\bigskip
Ignorability: include covariates to get
conditional ignorability
\pause
\begin{eqnarray*}
(Y(1), Y(0)) \bot T | X
\end{eqnarray*}
\pause
Treatment assignment is independent of the outcomes ($Y$) given
covariates $X$.
\end{frame}

\begin{frame}
Problems:
\pause
\bigskip
\begin{itemize}
\item SUTVA assumption
\pause
\bigskip
\item Omitted variable bias
\pause
\begin{itemize}
\item Don't include all the variables that makes treatment assignment independent of $Y$.
\end{itemize}
\pause
\bigskip
\item Model Dependence
\pause
\begin{itemize}
\item We try to alleviate the curse of dimensionality and problem of
continuous covariates by specifying a model.
\pause
\item Estimates of ATE or ATT may differ depending on the model you specify.
\end{itemize}
\end{itemize}
\end{frame}

\begin{frame}
\frametitle{Matching to Ameliorate Model Dependence}
\pause
If we had pairs of observations that had the exact same covariate
values (perfect {\bf balance}) and differed only on treatment assignment, then we would have
perfect conditional ignorability (assuming no omitted variable
bias).\\
\pause
\bigskip
Then we will get the same results regardless of the model.\\
\pause
\bigskip
Matching is a method of trying to achieve better balance on covariates
and reduce model dependence. \\
\pause
\bigskip
Goal: {\bf BALANCE} on covariates
\end{frame}

\begin{frame}
\frametitle{Matching on Propensity Scores}
\pause
Suppose each observation has some true probability of receiving the
treatment.
\pause
\begin{itemize}
\item A doctor examines a patient and has a probability of
giving the patient a drug, depending on the patient's age, health, etc.
\end{itemize}
\bigskip
\pause
The probability of receiving the treatment is the {\bf propensity
score}. \\
\pause
\bigskip
We don't know the true propensity score but we can estimate it for
each observation with a regression of $T$ on $X$ \pause (assuming we have the
right set of $X$ that went into the decision for assigning
treatment).\\
\pause
\bigskip
Then we match an observation that received treatment with an
observation with a similar propensity score that received control.\\
\pause
\bigskip
Since we don't have the true propensity scores, we need to check for
balance on our covariates at the end.
\end{frame}

\begin{frame}
\frametitle{Other Matching Algorithms}
\pause
Matching on propensity scores is only one way of matching. \\
\pause
\bigskip
Other ways include:
\pause
\begin{itemize}
\item Matching on Mahalanobis distances
\pause
\item Genetic Matching
\pause
\item Coarsened Exact Matching (CEM)
\end{itemize}
\end{frame}

\begin{frame}
\frametitle{Instrumental Variables}
\pause
Goal: Estimate Causal Effects\\
\bigskip
\pause
Problem in Observational Data: Non-ignorability of treatment
assignment \pause (and SUTVA) \\
\pause
\bigskip
Solution so far: Include covariates and match \\
\pause
\bigskip
Another solution: {\bf Instrumental Variables} \\
\pause
\bigskip
The idea: Find an instrument $Z$ that is randomly assigned (or
assignment is ignorable) and that affects $Y$ only through $T$. \\
\pause
\bigskip
Example: $Y$ = post-Vietnam War civilian mortality; \pause $T$ = serving in the
military during Vietnam War; \pause $Z$ = draft lottery
\end{frame}

\begin{frame}
\frametitle{The Potential Outcomes Approach}
\pause
Assumptions:
\pause
\bigskip
\begin{enumerate}
\item SUTVA:
\pause 
$Z_i$ does not affect $T_j$ and $Y_j$ and $T_i$ does not
affect $Y_j$ for all $i \neq j$ (non-interference) and there is no
variation in the treatment or the instrument.
\pause
\small
\begin{figure}[!htp]
\caption{SUTVA Assumption implies absence of dotted arrows.}
\centerline{
\xymatrix{
Z_i \ar[r] \ar@{.>}[rdd] \ar@{.>}[rrdd] & T_i \ar[r] \ar@{.>}[rdd] & Y_i\\
& & \\
Z_j \ar[r] & T_j \ar[r] & Y_j\\
}
}
\end{figure}
\pause
\bigskip
\tiny
{\it Example: The veteran status of any man at risk of being drafted
in the lottery was not affected by the draft status of others at risk
of being drafted, and, similarly, that the civilian mortality of any
such man was not affected by the draft status of others.}
\normalsize
\end{enumerate}
\end{frame}

\begin{frame}
\begin{itemize}
\item [2.] Random (Ignorable) Assignment of the Instrument $Z$ \\
\pause
\tiny
{\it Example: Assignment of draft status was random.}
\normalsize
\bigskip
\pause
\item [3.] Exclusion Restriction: \pause Any effect of $Z$ on $Y$ must
be via an effect of $Z$ on $T$.
\pause
\small
\begin{figure}[!htp]
\caption{Exclusion assumption implies absence of dotted arrow.}
\centerline{
\xymatrix{
Z_i \ar[r] \ar@(dr,dl) @{.>}[rr]& D_i \ar[r] & Y_i\\
}
}
\end{figure}
\pause
\tiny
{\it Example: Civilian mortality risk was not affected by draft status once
veteran status is taken into account.}
\normalsize
\bigskip
\item [4.] Nonzero Average Causal Effect of $Z$ on $T$.\\
\pause
\tiny
{\it Example:  Having a low lottery number increases the average probability of service.}
\normalsize
\bigskip
\pause
\item [5.] Monotonicity: \pause No Defiers\\
\pause
\tiny
{\it Example: There is no one who would have served if given a high lottery number, but not if given a low lottery number.}
\normalsize
\end{itemize}
\end{frame}

\begin{frame}
If all the assumptions hold, then the {\bf Local Average Treatment
Effect (LATE)} of $T$ on $Y$ is 
\begin{eqnarray*}
\mathrm{LATE} = \frac{\mathrm{Effect \; of \; Z \; on \;
Y}}{\mathrm{Effect \; of \; Z \; on \; T}} 
\end{eqnarray*}
\pause
It is only a local average treatment effect because it's the effect of
$T$ on $Y$ for the subpopulation of compliers, and not the whole population.
\end{frame}

\begin{frame}
Angrist, Joshua D., Guido W. Imbens and Donald B. Rubin. 1996. ``Identification of Causal Effects
Using Instrumental Variables.'' {\it Journal of the American
Statistical Association} 91(434):444-455. \\
\bigskip
Describes instrumental variables in more detail and compares it to the
econometric treatment.
\end{frame}

\end{document}
