\documentclass[handout]{beamer}

\usepackage{bm}
\usepackage{multicol}

\title{Learning \LaTeX}
\author{Patrick Lam}
\date{}

\newcommand{\red}{\color{red}}
\newcommand{\black}{\color{black}}


\begin{document}
\maketitle

\begin{frame}
\frametitle{setting up}
\pause
\begin{enumerate}
\item download a TeX distribution (MiKTeX, MacTeX, etc.)
\item download an editor (Texmaker, WinEDT, XEmacs, etc.)
\item start a .tex file in editor
\item work only in the .tex file
\end{enumerate}
\end{frame}

\begin{frame}[fragile]
\begin{center}
Let {\tt <>} denote things you fill in, but without the {\tt <>}. \\

\pause 
\bigskip
For example, for {\tt <myname>}, I would write {\tt Patrick}. \\
\end{center}
\end{frame}

\begin{frame}
\begin{center}
Example code is in \red{red}
\end{center}
\end{frame}

\begin{frame}[fragile]
\frametitle{commands}
\pause
You have to tell \LaTeX\hspace{1pt} to do everything with commands, which always begin with $\backslash$: \\

\pause
\bigskip

\red
\begin{verbatim}
\<command>
\<command>{<something>}
\<command>[<options>]{<something>}
\end{verbatim}
\black 

\end{frame}

\section{Getting Started}
\begin{frame}[fragile]
\frametitle{the bare minimum}
\pause
\red
\begin{verbatim}
\documentclass[<options>]{<class>}

\begin{document}

My text here!

\end{document}

\end{verbatim}
\black 
\pause
{\tt article}, {\tt beamer} $\in$ {\tt <class>} \pause \\
translation: some {\tt class} types include {\tt article} and {\tt beamer}\\
\pause
\bigskip
For almost everything, we will be using {\tt article}.
\end{frame}

\section{Compiling}
\begin{frame}[fragile]
\frametitle{.tex -$>$ something readable }
On your .tex file, compile using LaTeX or PDFLaTeX (usually buttons or commands on your editor).
\pause
\bigskip
\begin{itemize}
\item .tex file -$>$ LaTeX -$>$ .dvi file 
\pause
\begin{itemize}
\item you can convert .dvi to .pdf with dvi2pdf or dvi2ps and ps2pdf
\end{itemize}
\medskip
\pause
\item .tex file -$>$ PDFLaTeX -$>$ .pdf file
\end{itemize}
\pause
\bigskip
Output files will be in the same directory as your .tex file. 
\end{frame}

\begin{frame}[fragile]
\begin{center}
Try compiling now! \\
\pause
\bigskip
\bigskip
If you get an error message, something is wrong in your code.\\
\pause
\bigskip
\bigskip
Compile often to catch errors before they pile up!
\end{center}
\end{frame}

\begin{frame}[fragile]
\begin{center}
Let's get started with more complicated stuff!
\end{center}
\end{frame}

\begin{frame}[fragile]
\frametitle{title, author, date}
Let's give our article a title, author, and date.
\pause
\red
\begin{verbatim}
\documentclass{article}

\title{This is my title}
\author{Patrick Lam}
\date{}

\begin{document}

My text here!

\end{document}
\end{verbatim}
\black 
\end{frame}

\begin{frame}[fragile]
\frametitle{packages}
Tell \LaTeX\hspace{1pt} to use some packages before beginning the document:
\pause
\red
\begin{verbatim}
\documentclass{article}

\usepackage{<package>}
\usepackage{<package>}

\title{This is my title}
\author{Patrick Lam}
\date{}

\begin{document}

My text here!

\end{document}

\end{verbatim}
\black 
\end{frame}

\begin{frame}[fragile]
Everything else from now on goes after {\tt $\backslash$begin\{document\}}.
\pause
\bigskip
\red
\begin{verbatim}
\begin{document}

Everything I write should go here.

\end{document}
\end{verbatim}
\black 
\end{frame}

\begin{frame}
\begin{center}
OK, let's try compiling what we have. \\
\pause
\bigskip
\bigskip
Where is our title, abstract and date?
\end{center}
\end{frame}

\begin{frame}[fragile]
We have to tell \LaTeX\hspace{1pt} to put it in our document with {\tt $\backslash$maketitle} or {\tt $\backslash$titlepage}:
\red
\pause
\begin{verbatim}
\begin{document}
\maketitle

My text here!

\end{document}
\end{verbatim}
\black 
\pause
\bigskip
Try compiling again!
\end{frame}

\section{Line Spacing}
\begin{frame}[fragile]
\frametitle{lines}
\LaTeX\hspace{1pt} is smart with line spacing. \\
\pause
\bigskip
To start a new paragraph, skip a line in your .tex file:
\pause
\red
\begin{verbatim}
Paragraph 1

Paragraph 2

Paragraph 3
\end{verbatim}
\black 

\end{frame}

\begin{frame}[fragile]
To end a line and start a new line, use $\backslash\backslash$:
\pause
\red
\begin{verbatim}
This is my line. \\
This is my new line.
\end{verbatim}
\black 
\end{frame}

\begin{frame}[fragile]
To put space between lines, use multiple {\tt $\backslash\backslash$, $\backslash$bigskip, $\backslash$medskip}, or {\tt $\backslash$smallskip}:
\pause
\red
\begin{verbatim}
This is my line. \\\\
This is my new line. \\
\bigskip
This is another line.
\end{verbatim}
\black 
\end{frame}

\begin{frame}[fragile]
To double space, use the {\tt setspace} package and the {\tt $\backslash$doublespacing} command:
\pause
\red
\begin{verbatim}
\documentclass{article}

\usepackage{setspace}

\begin{document}

\doublespacing

This document is now doublespaced.\\
See. 

\end{document}
\end{verbatim}
\black 
\end{frame}


\begin{frame}
\frametitle{full page}
\begin{itemize}
\item by default, \LaTeX\hspace{1pt} uses large margins and single spacing as the optimal format
\pause
\item to get regular margins, use the {\tt fullpage} package
\end{itemize}
\end{frame}

\section{Sectioning}
\begin{frame}[fragile]
\frametitle{sectioning}
\LaTeX\hspace{1pt} can divide out document into sections with titles using $\backslash$section (with numbers) or $\backslash$section* (without numbers). 
\pause
\red
\begin{verbatim}
\section{My First Section Title} 
Text here

\section{My Second Section Title}
More text here.
\end{verbatim}
\black
\pause
\bigskip
$\backslash$subsection and $\backslash$subsubsection also available.
\pause
\bigskip
Try and compile!
\end{frame}

\section{Environments}
\begin{frame}[fragile]
\frametitle{environments}
\pause
Think of environments as creating space in your document for certain activities.  \pause Environments must always begin and end. \\

\red
\begin{verbatim}
\begin{<environment>}

Stuff here!

\end{<environment>}
\end{verbatim}
\black 
\pause
{\tt document}, {\tt verbatim}, {\tt equation}, {\tt eqnarray}, {\tt table}, {\tt tabular}, {\tt figure}, {\tt center}, {\tt itemize}, {\tt enumerate} $\in$ {\tt <environment>}

\end{frame}

\section{Lists}
\begin{frame}[fragile]
\frametitle{lists}
Let's create a list environment with {\tt itemize} or {\tt enumerate}:
\pause
\red
\begin{verbatim}
My favorite drinks:

\begin{itemize}
\item Barq's Root Beer
\item Dr. Pepper
\item Orange Soda
\end{itemize}
\end{verbatim}
\pause
\bigskip
\black 
What's the difference between the two?
\end{frame}


\section{Math}
\begin{frame}
\frametitle{math mode}
One of the major advantages of \LaTeX\hspace{1pt} is in typing math. \\
\pause
\bigskip
There are many ways to go to math mode.  I like:
\pause
\bigskip
\begin{itemize}
\item \$ for inline math \pause
\item {\tt eqnarray} environment for centered math and equations
\end{itemize}
\bigskip
\pause
Others include:
\bigskip
\begin{itemize}
\item \$\$
\item {\tt equation} environment for one equation only
\item {\tt displaymath} environment
\end{itemize}
\end{frame}

\begin{frame}[fragile]
For math within a line such as $\alpha$ or $5 > 4$, enclose the math in dollar signs: \\
\pause
\bigskip
\red
{\tt
For math within a line such as \$$\backslash$alpha\$ or \$5 $>$ 4\$, enclose the math in dollar signs:
}
\black 
\end{frame}

\begin{frame}[fragile]
For equations or centered math, use the {\tt eqnarray} environment.  \pause You can align using \&.
\pause
\begin{eqnarray}
5+4 &=& 9 \\
3+2 &=& 5
\end{eqnarray}
\pause
\red
\begin{verbatim}
\begin{eqnarray}
5+4 &=& 9 \\
3+2 &=& 5
\end{eqnarray}
\end{verbatim}
\black 
\end{frame}

\begin{frame}[fragile]
\frametitle{other math stuff}
\pause
\begin{itemize}
\item Greek symbols are intuitive: \red {\tt $\backslash$delta $\backslash$Delta} \black 
\pause
\item scripting is easy: $x_{ij} + x^2$ is \red \verb x_{ij}+x^2 \black 
\pause
\item bolding math is with \red \verb \mathbf{<text>} \black \hspace{1pt} for non-Greek symbols and \red \verb \bm{<symbol>} \black  \hspace{1pt} in the {\tt bm} package for Greek symbols
\pause
\item fractions are \red \verb \frac{<numerator>}{<denominator>} \black
\pause
\item square roots are \red \verb \sqrt{<number>} \black 
\pause
\item many many other things you can do 
\pause
\item use google to search!
\end{itemize}
\end{frame}

\begin{frame}[fragile]
Try typing the following equations:
\begin{eqnarray*}
\bm{\hat{\beta}} &=& (\mathbf{X'X})^{-1} \mathbf{X'y} \\
f(x) &=& \frac{1}{\sigma \sqrt{2 \pi}} \exp \left( -\frac{(x-\mu)^2}{2\sigma^2} \right)
\end{eqnarray*}
\end{frame}

\section{Tables}
\begin{frame}[fragile]
\frametitle{tables}
Tables are created using the {\tt tabular} environment.
\pause
\begin{multicols}{2}
\red
\begin{verbatim}
\begin{tabular}{l|c|r}
year & country & leader \\
\hline
2009 & US & Obama \\
2009 & UK & Brown \\
\hline
\end{tabular}
\end{verbatim}
\black 
\newpage
\pause
\bigskip
\bigskip
\bigskip
\begin{tabular}{l|c|r}
year & country & leader \\
\hline
2009 & US & Obama \\
2009 & UK & Brown \\
\hline
\end{tabular}
\end{multicols}
\pause
\begin{itemize}
\item \{\} arguments after {\tt $\backslash$begin\{tabular\}} specify number of columns and alignment
\pause
\begin{itemize}
\item use $\mid$ for vertical dividers.
\end{itemize}
\pause
\item \& is used to specify column breaks
\pause
\item use {\tt $\backslash$hline} for horizontal lines  
\end{itemize}
\end{frame}

\begin{frame}[fragile]
Enclose the table in a {\tt table} environment for captions and placement options ({\tt [!htp]}).
\pause
\begin{multicols}{2}
\red
\begin{verbatim}
\begin{table}[!htp]
\caption{table caption} 
\begin{tabular}{l|c|r}
year & country & leader \\
\hline
2009 & US & Obama \\
2009 & UK & Brown \\
\hline
\end{tabular}
\end{table}
\end{verbatim} 
\black 
\newpage
\pause
\begin{table}[!htp]
\caption{table caption} 
\begin{tabular}{l|c|r}
year & country & leader \\
\hline
2009 & US & Obama \\
2009 & UK & Brown \\
\hline
\end{tabular}
\end{table}
\end{multicols}
\pause
Try creating your own tables!
\end{frame}

\section{Graphics}
\begin{frame}[fragile]
\frametitle{graphics}
Include graphics (plots, pictures, etc.) with the {\tt graphicx} package and {\tt $\backslash$includegraphics}.
\pause
\red
\begin{verbatim}
\includegraphics[<options>]{myfilename.jpg}
\end{verbatim}
\black 
\pause
\begin{itemize}
\item adjust the size in options (i.e. {\tt [scale = .5]} or {\tt [width = 2in, height=2in]})
\pause
\item for compiling with PDFLaTeX, file must be .jpg, .pdf, or .png
\pause
\item for compiling with LaTeX, file must be .ps or .eps
\pause
\item embed graphic in a {\tt figure} environment for captions and placement (similar to {\tt table})
\end{itemize}
\end{frame}

\section{Floats}
\begin{frame}[fragile]
\frametitle{floats}
The {\tt table} and {\tt figure} environments are known as \textit{floats}, which are objects that cannot be broken up into multiple pages. \\
\bigskip
\pause

Floats have three qualities that you should know:
\pause
\begin{itemize}
\item placement
\item caption
\item labels and cross-referencing
\end{itemize}

\end{frame}

\begin{frame}[fragile]
placement: 
\pause
\begin{itemize}
\item h: here
\item t: top of page
\item b: bottom of page
\item p: special page for floats
\item !: override \LaTeX\hspace{1pt} defaults
\end{itemize}
\pause
\bigskip
\LaTeX\hspace{1pt} will try to accomodate in order.
\pause
\red
\begin{verbatim}
\begin{figure}[!htp]
\includegraphics{myfilename.jpg}
\end{figure}
\end{verbatim}
\black 

\end{frame}

\begin{frame}[fragile]
caption: 
\pause
\begin{itemize}
\item use the {\tt caption} command
\pause
\item can be placed below or above
\end{itemize}
\pause
\red
\begin{verbatim}
\begin{figure}[!htp]
\includegraphics{myfilename.jpg}
\caption{my caption here}
\end{figure}
\end{verbatim}
\black 
\end{frame}

\begin{frame}[fragile]
labels and cross-referencing:
\pause
\begin{itemize}
\item \LaTeX\hspace{1pt} will keep an internal numbering of the floats (separate counts for tables and figures)
\pause
\item if you have captions, \LaTeX\hspace{1pt} will number your floats for you (i.e. Figure 1: my caption)
\pause
\item use the {\tt $\backslash$label} and {\tt $\backslash$ref} commands to refer to the float number in your text
\end{itemize}
\end{frame}

\begin{frame}[fragile]
\red
\begin{verbatim}
\begin{figure}[!htp]
\includegraphics{myfilename.jpg}
\caption{my caption here}
\label{<key>}
\end{figure}

In Figure \ref{<key>}, we can see that...
\end{verbatim}
\black 
\pause
\begin{itemize}
\item {\tt $\backslash$label} must be below {\tt $\backslash$caption}
\pause
\item if figure is Figure 1, then text will show "In Figure 1, we can see that..."
\pause
\item {\tt <key>} is any word, phrase, alphanumeric indicator you want
\pause
\item may have to compile more than once to get the references to show up correctly
\end{itemize}
\end{frame}

\section{Verbatim}
\begin{frame}[fragile]
\frametitle{verbatim}
One particular environment useful for pasting R code is the {\tt verbatim} environment.
\pause
\red
\begin{verbatim}

\begin{verbatim}
LaTeX will copy everything in the verbatim 
environment exactly, rather than interpret it as code.  
For example, \begin{itemize} 
here is typed out exactly rather than beginning a list.
\end{ verbatim}

\end{verbatim}
\black 
\end{frame}

\section{Fonts}
\begin{frame}
\frametitle{fonts}
\pause
\begin{itemize}
\item define the font size for the document in as an option in {\tt $\backslash$documentclass}: \pause \red {\tt $\backslash$documentclass[12pt]\{article\}} \black 
\pause
\item change font size in text using \tiny {\tt $\backslash$tiny}, \scriptsize{\tt $\backslash$scriptsize}, \footnotesize{\tt $\backslash$footnotesize}, \small{\tt $\backslash$small}, \normalsize{\tt $\backslash$normalsize}, \large{\tt $\backslash$large},
\Large{\tt $\backslash$Large}, \LARGE{\tt $\backslash$LARGE}, \huge{\tt $\backslash$huge}, \Huge{\tt $\backslash$Huge} \normalsize
\pause
\item \textcolor{purple}{colored fonts} using {\tt color} package
\pause
\item fonts in math mode are different
\end{itemize}
\end{frame}


\section{Footnotes}
\begin{frame}[fragile]
\frametitle{footnotes}
use the {\tt $\backslash$footnote} command\footnote{like this}
\pause
\red
\begin{verbatim}
use the {\tt $\backslash$footnote} 
command\footnote{like this}
\end{verbatim}
\black 
\end{frame}


\section{Bibliography}
\begin{frame}
\frametitle{bibliography}
Compiling a bibliography is simple in \LaTeX\hspace{1pt}.
\pause
\begin{itemize}
\item need to use the {\tt natbib} package
\pause
\item need a separate file with .bib extension in the same directory 
\pause
\item in your .tex file, need the line \red {\tt $\backslash$bibliography\{<name of bib file w/o .bib extension>\}} \black 
\pause
\item also in your .tex file, need the line \red {\tt $\backslash$bibliographystyle\{<name of a style>\}} \black 
\pause
\begin{itemize}
\item styles may be downloaded, including an apsr style
\end{itemize}
\end{itemize}
\end{frame}

\begin{frame}[fragile]
in .bib file, for each reference, need something like 
\red
\begin{verbatim}
@Article{<key>,
author = {},
title = {},
journal = {},
year = {},
}
\end{verbatim}
\black 
\pause
\begin{itemize}
\item {\tt <key>} is a unique identifier for each reference that you will refer to in your .tex file
\pause
\item your editor should have these fields preset somewhere
\end{itemize}
\pause
\bigskip
in .tex file, use {\tt $\backslash$citep\{<key>\}}, {\tt $\backslash$citet\{<key>\}}, or {\tt $\backslash$citen\{<key>\}} where you want to reference a certain work
\end{frame}

\begin{frame}
when compiling
\pause
\begin{enumerate}
\item LaTeX or PDFLatex once
\pause
\item BibTeX once (should be a button or compiling command)
\pause
\item LaTeX or PDFLatex once
\pause
\item LaTeX or PDFLatex once more
\end{enumerate}
\end{frame}

\section{Presentations}
\begin{frame}
\frametitle{presentations}
Powerpoint-like presentations can be done using a document class called {\tt beamer} (may need to be downloaded).
\pause
\begin{itemize}
\item can choose a color theme (default theme used here)
\pause
\item same basic code
\pause
\item each slide is a {\tt frame} environment
\pause
\item use {\tt $\backslash$pause} between clicks
\end{itemize}
\end{frame}

\section{The End}
\begin{frame}
\begin{center}
\textbf{anything} can be done using \LaTeX\hspace{1pt} with the right package
\end{center}
\end{frame}

\begin{frame}
\begin{center}
you will learn much much much more along the way
\end{center}
\end{frame}

\begin{frame}
\begin{center}
google is your friend
\end{center}
\end{frame}

\begin{frame}
\frametitle{exercises}
Include the following into one .pdf document:
\begin{itemize}
\item the following equations:
\begin{eqnarray*}
p(\theta | \bm{y}) &\propto& p(\bm{y} | \theta) p(\theta)\\
&=& \exp \left[ -\frac{1}{2 \sigma^2
\tau_0^2} \left(\theta^2\left(\sigma^2 +
\tau_0^2 n \right) - 2 \theta \left( \mu_0
\sigma^2 + \tau_0^2 n \bar{y} \right)  \right) \right] \\
\end{eqnarray*}
\item a $5 \times 3$ table of 5 Harvard political scientists (or other academics) with their names, field or subfield, and office room number
\item a picture of your favorite celebrity with an appropriate caption
\item a list of your 3 favorite things about Harvard so far
\end{itemize}
\end{frame}
\end{document}