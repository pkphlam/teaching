\documentclass[10pt,a4paper]{article}
\usepackage[latin1]{inputenc}
\usepackage{amsmath}
\usepackage{amsfonts}
\usepackage{amssymb}
\usepackage{fullpage}
\usepackage{color}
\usepackage{Sweave}
\author{Patrick Lam}

\parindent=0in
\newcommand{\red}{\color{red}}
\newcommand{\black}{\color{black}}
\begin{document}

\begin{center}
\begin{Large}Week 4 Problems\end{Large}
\end{center}
\bigskip
\begin{enumerate}

\item (Adapted from Gelman 2.10) Suppose there are $N$ cable cars in San Francisco, numbered sequentially from 1 to $N$.  You see a cable car at random; it is numbered 203.  You wish to estimate $N$. \\

Assume your prior distribution on $N$ is geometric with mean 100; that is,
\begin{eqnarray*}
p(N) = (1/100)(99/100)^{N-1}, \; \mathrm{for} \; N = 1,2,\dots
\end{eqnarray*}

\begin{itemize}
\item[a)] What is your posterior distribution for $N$ up to a constant of proportionality?   

\medskip \red
\begin{eqnarray*}
p(N|X) &\propto& p(X|N) p(N) \\
&=& \frac{1}{N} \left( \frac{1}{100} \right) \left( \frac{99}{100} \right)^{N-1} \, \mathrm{for} \, N\ge 203\\
&\propto& \frac{1}{N} \left( \frac{99}{100} \right)^{N-1} \, \mathrm{for} \, N\ge 203
\end{eqnarray*}
\medskip \black 

\item[b)] Find the posterior mean and standard deviation by approximating the normalizing constant in R (without simulating).  \\

\medskip \red 
The posterior is 
\begin{eqnarray*}
p(N|X) &=& \frac{p(X|N)p(N)}{p(X)} \\
&=& \frac{1}{p(X)} \frac{1}{N} \left( \frac{99}{100} \right)^{N-1} \, \mathrm{for} \, N\ge 203
\end{eqnarray*}
The normalizing constant can be approximated by
\begin{eqnarray*}
p(X) \approx \sum_{N=203}^{10000} \frac{1}{N} \left( \frac{99}{100} \right)^{N-1}
\end{eqnarray*}
The posterior mean and standard deviation are then approximated by
\begin{eqnarray*}
E[p(N|X)] &\approx& \sum_{N=203}^{10000} N \frac{\frac{1}{N} \left( \frac{99}{100} \right)^{N-1}}{p(X)} = \sum_{N=203}^{10000} \frac{\left( \frac{99}{100} \right)^{N-1}}{p(X)}\\
sd[p(N|X)] &\approx& \sqrt{\sum_{N=203}^{10000} (N-E[p(N|X)]) \frac{\frac{1}{N} \left( \frac{99}{100} \right)^{N-1}}{p(X)}}
\end{eqnarray*}
\small
\begin{Schunk}
\begin{Sinput}
> N <- 203:10000
> p.x <- sum((1/N) * (99/100)^(N - 1))
> E.N <- sum(((99/100)^(N - 1))/p.x)
> E.N
\end{Sinput}
\begin{Soutput}
[1] 279.0885
\end{Soutput}
\begin{Sinput}
> sd.N <- sqrt(sum((N - E.N)^2 * ((1/N) * (99/100)^(N - 1))/p.x))
> sd.N
\end{Sinput}
\begin{Soutput}
[1] 79.96458
\end{Soutput}
\end{Schunk}
\medskip \black \normalsize

\item[c)] Now find the posterior mean and standard deviation by simulation in R.  Are your answers similar to those in b)?

\medskip \red \small
\begin{Schunk}
\begin{Sinput}
> N <- 203:10000
> n.sim <- 10000
> unnormal.post <- (1/N) * (99/100)^(N - 1)
> post.draws <- sample(N, size = n.sim, prob = unnormal.post, 
+     replace = T)
> mean(post.draws)
\end{Sinput}
\begin{Soutput}
[1] 280.2904
\end{Soutput}
\begin{Sinput}
> sd(post.draws)
\end{Sinput}
\begin{Soutput}
[1] 81.43097
\end{Soutput}
\end{Schunk}
\medskip \black \normalsize
\end{itemize}

\bigskip

\item (adapted from Gelman 3.2) On September 25, 1988, the evening of a Presidential campaign debate, ABC News conducted a survey of registered voters in the United States; 639 persons were polled before the debate, and 639 different persons were polled after.  \\

\begin{table}[!htp]
\begin{center}
\begin{tabular}{ccccc}
Survey & Bush & Dukakis & No opinion/other & Total \\
\hline
pre-debate & 294 & 307 & 38 & 639 \\
post-debate & 288 & 332 & 19 & 639 \\
\end{tabular}
\end{center}
\end{table}

Assume the surveys are independent simple random samples from the population of registered voters.  Model the data with two different multinomial distributions.
\begin{enumerate}
\item[a)] What is the posterior probability that support for Bush increased between the two surveys?

\medskip \red \small
\begin{Schunk}
\begin{Sinput}
> library(MCMCpack)
> n.sim <- 10000
> y.1 <- c(294, 307, 38)
> y.2 <- c(288, 332, 19)
> alpha.0 <- c(1, 1, 1)
> post.1 <- rdirichlet(n.sim, alpha = y.1 + alpha.0)
> post.2 <- rdirichlet(n.sim, alpha = y.2 + alpha.0)
> mean(post.2[, 1] > post.1[, 1])
\end{Sinput}
\begin{Soutput}
[1] 0.3594
\end{Soutput}
\end{Schunk}
\medskip \black \normalsize 

\item[b)] Of the voters who had a preference for either Bush or Dukakis, what is the posterior probability that there was a shift toward Bush between the two surveys?

\medskip \red \small
\begin{Schunk}
\begin{Sinput}
> a <- b <- 1
> y.bd.1 <- c(294, 307)
> y.bd.2 <- c(288, 332)
> post.bd.1 <- rbeta(n.sim, y.bd.1[1] + a, y.bd.1[2] + b)
> post.bd.2 <- rbeta(n.sim, y.bd.2[1] + a, y.bd.2[2] + b)
> mean(post.bd.2 > post.bd.1)
\end{Sinput}
\begin{Soutput}
[1] 0.1936
\end{Soutput}
\end{Schunk}
\medskip \black \normalsize

\end{enumerate}

\bigskip

\item Suppose we have $n$ observations that follow a Normal distribution with a common mean $\mu$ and variance $\sigma^2$.  Also, suppose that we know the mean of the data, but want to learn about the variance of the data.  Find the posterior distribution of the variance $\sigma^2$ given an Inverse-gamma prior.  Specifically, find $p(\sigma^2 | \mathbf{y})$ given
\begin{eqnarray*}
Y_i &\sim& \text{N}(\mu, \sigma^2) \\
\sigma^2 &\sim& \text{Inv-gamma}(\alpha, \beta)
\end{eqnarray*}

\medskip \red
\begin{eqnarray*}
  p(\sigma^2|\mathbf{y}) &\propto& \prod_{i=1}^n \frac{1}{\sqrt{2\pi\sigma^2}} \exp
  \left( -\frac{(y_i - \mu)^2}{2\sigma^2} \right) \times 
  \frac{\beta^{\alpha}}{\Gamma(\alpha)} (\sigma^2)^{-(\alpha +
  1)} e^{-\beta / \sigma^2} \\
&\propto& \prod_{i=1}^n (\sigma^2)^{-\frac{1}{2}} \exp
  \left( -\frac{(y_i - \mu)^2}{2\sigma^2} \right)  (\sigma^2)^{-(\alpha +
  1)} e^{-\beta / \sigma^2} \\
&=& (\sigma^2)^{-\frac{n}{2}} \exp
  \left( -\frac{\sum_{i=1}^n (y_i - \mu)^2}{2\sigma^2} \right)  (\sigma^2)^{-(\alpha +
  1)} e^{-\beta / \sigma^2} \\
 &=& (\sigma^2)^{-(\alpha + \frac{n}{2}+1)} \exp
  \left[ -\left( \frac{\beta}{\sigma^2} + \frac{\sum_{i=1}^n (y_i -
        \mu)^2}{2\sigma^2} \right) \right]  \\
  &=& (\sigma^2)^{-(\alpha + \frac{n}{2}+1)} \exp
  \left[ -\left( \frac{2\beta + 2 \left( \frac{\sum_{i=1}^n (y_i -
        \mu)^2}{2} \right)}{2\sigma^2} \right) \right]  \\
&=& (\sigma^2)^{-(\alpha + \frac{n}{2}+1)} \exp
  \left[ -\left( \frac{\beta + \frac{\sum_{i=1}^n (y_i -
        \mu)^2}{2} }{\sigma^2} \right) \right]  \\
\end{eqnarray*}
The posterior is
an Inv-gamma$\left( \alpha+ \frac{n}{2}, \beta + \frac{\sum_{i=1}^n (y_i -
        \mu)^2}{2}\right)$ distribution.
\medskip \black 
\end{enumerate}

\end{document}





