\documentclass[10pt,a4paper]{article}
\usepackage[latin1]{inputenc}
\usepackage{amsmath}
\usepackage{amsfonts}
\usepackage{amssymb}
\usepackage{fullpage}
\usepackage{color}
\author{Patrick Lam}

\parindent=0in
\newcommand{\red}{\color{red}}
\newcommand{\black}{\color{black}}
\begin{document}

\begin{center}
\begin{Large}Week 3 Problems\end{Large}
\end{center}
\bigskip
\begin{enumerate}

\item (Adapted from Jackman 2.3) Three polls are fielded on the weekend prior to an election, each using simple random sampling, producing the following estimates of vote intention for the incumbent candidate.
\medskip

\begin{center}
\begin{tabular}{lcc}
\hline
Poll & Support (\%) & Sample size \\
\hline
A & 52 & 350 \\
B & 49 & 500 \\
C & 55 & 600 \\
\hline
\end{tabular}
\end{center}
\medskip
\begin{itemize}
\item[a)] Let the proportion of the electorate intending to vote for the incumbent be $\theta$.  Write out a plausible statistical model for each poll's results in which $\theta$ appears as a parameter.  Specifically, specify a likelihood and prior for your model. \\
\item[b)] Suppose your have no prior information over $\theta$.  Taking each poll separately, what is the posterior density for $\theta$ given the information in each poll?  Find posterior means, standard deviations, and credible intervals using R. \\
\item[c)] Now suppose you observe each poll in order sequentially starting with A.  Update your posterior with each new poll and show how your estimates change. \\
\item[d)] Now suppose you observe all the polls simultaneously such that you can pool the results together.  Analytically show that your posterior is equivalent to the posterior from c).\\
\item[e)] Suppose the electorate is composed of 3,000 voters who must all cast a vote.  Using simulation and your posterior from d), predict the probability that the incumbent will win the election given majority rule. \\
\item[f)] Now suppose that the electoral rule dictates that a candidate can only win with a supermajority (55\%).  What is the probability that the incumbent will win the election given this new rule?
\end{itemize}


\bigskip

\item A prior is considered flat or uninformative if it is proportional to a constant.
\begin{eqnarray*}
p(\theta) \propto c
\end{eqnarray*}
For a model with a Binomial($n$, $\theta$) likelihood,
\begin{itemize}
\item[a)] show that a Beta(1,1) prior on $\theta$ is uninformative. \\
\item[b)] show that if we transform the parameter such that we have a new parameter $\Theta = \exp(\theta)$, a Beta(1,1) prior on $\theta$ is no longer uninformative on $\Theta$. 
\end{itemize}

\bigskip

\item (From Jackman Example 2.9) Geddes (1990) examined the relationship between foreign threat and social revolution in Latin America using data that generated the following 2-by-2 table.

\medskip
\begin{center}
\begin{tabular}{lcc}
\hline 
& Revolution & No revolution \\
\hline
Defeated and invaded or lost territory & 1 & 7 \\
Not defeated for 20 years & 2 & 74 \\
\hline
\end{tabular}
\end{center}
\medskip

Each observation is a 20-year period for each Latin American country, with any given country contributing multiple observations.  Propose a model to assess the hypothesis that suffering a military loss within a 20-year period increases the chances of revolution within a country.  What is the probability that military defeat is positively associated with revolution?

\end{enumerate}

\end{document}





